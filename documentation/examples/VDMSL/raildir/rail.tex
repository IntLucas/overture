\documentclass[11pt]{article}
\usepackage{durnew}
\usepackage{vdmsl-2e}
\usepackage{generated/latex/overture}
\usepackage{graphicx}

% Page usage. Big A4 size.
\oddsidemargin 1pt
\evensidemargin 1pt
\marginparwidth 30pt
\topmargin 1pt
% \headheight 12pt
% \headsep 8pt
\footskip 24pt
\textheight 650pt
\textwidth 460pt
% End of page usage.


\itletters

\parindent=0pt
\parskip=.5\baselineskip


\title{Modelling Railway Interlocking Systems\thanks{This work is supported by the
    Danish State Railways, by the project Mathematical
    Modelling of Computer Based Systems (MMaDS) at the Technical University of
    Denmark, and by the Commission of the European Communities (CEC),
    ESPRIT, Basic Research Action
    proj.\ no.\ 7071: ``ProCoS II: Provably Correct Systems''.}}
\author{Kirsten Mark Hansen\thanks{Department of Computer Science, Technical University
of Denmark Build. 344, DK-2800 Lyngby, Denmark, and
Danish State Railways, Infrastructure Management-Consult, Signalling
Safety Assessment, 
S\o lvgade 40, opg.\ F, 4.sal, DK-1349 K\o benhavn K, Denmark}}

\date{\today}

\begin{document}

\maketitle

\begin{abstract}
In this report we present a model of interlocking systems, and
describe how the model may be validated by simulation. Station
topologies are modelled by graphs in which the nodes denote track
segments, and the edges denote connectivity for train traffic. Points
and signals are modelled by annotations on the edges, thereby
restricting the driving possibilities. We define the safe station
states as predicates on the graph, and present a first step towards an
implementation of
these predicates. Both the model of station topologies, the safety
requirements, and the implementation are
validated using the VDM tool-box. The model development
illustrates how concepts may be captured and validated for a
non-trivial system.

\paragraph{CR Categories:} D.2.1, D.2.10, D.3.1, J.7
\paragraph{CR General Terms:} Design. 
\paragraph{Keywords:} Mathematical modelling, application of formal
methods, VDM-SL, stepwise
development, model validation, simulation, VDM tool-box, 
railway interlocking.
\end{abstract}


\section{Introduction}

The task of an
interlocking system is primarily to prevent trains from colliding, and
derailing, while at the same time allowing train movements. The Danish State
Railways
(DSB) has built and used interlocking systems for the past 150 years. The first
interlocking systems were pure mechanical systems, but as electricity became
common, the systems have developed through
electro mechanical to relay based systems and more recently to
computer based systems. Interlocking systems are safety
critical systems, so there is a need to ensure that the system prevents
dangerous
situations. Therefore DSB recently
decided to investigate the possibilities of using
formal software development for interlocking systems. This report presents
some of the
results of this work, which has been done as a cooperation between DSB and
the Department of Computer Science at the Technical University of
Denmark. Parts of this report have previously been published
in \cite{nsdcs94}, and in  \cite{fme94}.

In making a
formal model of interlocking systems,
we encountered a number of
difficulties, which seem to be general problems in
system modelling:

\begin{itemize}
\item no overall system requirements

\item a lack of precise concept definitions

\item the persons who develop
  interlocking systems had no knowledge of formal methods

\item we had no knowledge of interlocking systems.
\end{itemize}

The missing overall system requirements was a major problem.
There were informal specifications like
\cite{kravspec}, but they only contained low
level requirements like: `if the switching of a
point is not terminated within 20 seconds, the setting of the train route is
interrupted'.  There were no requirements stating the
purpose
of an interlocking system, and it turned out that
defining the overall task of an interlocking system was not a simple task.
One requirement could for instance be that an interlocking system
should prevent train
collision, but this is a matter for discussion, as joining
two trains into one strictly speaking is a train collision.

The second point was a lack of precise concept definitions.
Some concepts were not defined at all, e.g.\ train collision, and for
other concepts, the definition has changed over time, e.g.\ naming of
point branches.

Having made a system model there are generally two equal important aspects one
would like to investigate:

\begin{enumerate}
\item Is the model an acceptable model of  reality?

\item How is the model implemented?
\end{enumerate}

In formal program development one tends to focus on
the second aspect, but due to the difficulties mentioned above, we decided
to take validation seriously. 
Having
investigated a number of validation techniques, e.g.\ interviews,
structured reading \cite{validate,king}, symbolic execution of
specifications \cite{kneuper}, and prototyping (simulation)
\cite{prototyping}, we decided primarily to use simulation.
There were several reasons for this decision. First simulation is a
well-established technique in conventional mathematical modelling.
Next we had used interviews to reveal the requirements and therefore,
we did not trust interviews to be the only validation technique.
Structured reading means checking that there are no missing
references in a document, that all functions used are defined etc.,
but in choosing an executable specification language, we get this for
free. Finally symbolic execution of specifications did not fit very
well to our application as we had a specification with a small state
space and only one operation. 
By writing the specifications in VDM-SL
\cite{vdmsl}, and by using the VDM tool-box \cite{toolbox}, we were
able to execute the specification and thereby validate it. 

We do not formalise a particular interlocking system, but focus on
the principles and concepts upon which such systems are built. We take
interlocking systems used by the
DSB as starting point. Interlocking systems from other
countries may differ from the systems described here.

Other models
for interlocking systems
have been made in Higher Order Logic \cite{wai,morley3}, in
CCS \cite{morley2}, using double node graphs \cite{markus2,markus1}, and by
modelling tracks as units with end points \cite{sorenprehn}.
Many researchers have furthermore used railway applications as case studies,
e.g.\
\cite{jus,broy}, but this work is not directly applicable to our work, 
as it  either presents toy examples, or examples which have nothing to
do with interlocking, like e.g. level crossings \cite{jus}.

The content of the report is as follows: In Section~\ref{station} we
develop a model of station topologies and define their 
state spaces. In
Section~\ref{invar}, we define the safety requirements which an
interlocking system should fulfill, and in Section~\ref{intpre} we
present an implementation of the requirements. We
conclude in Section~\ref{concl} by
discussing the model development and executable
specifications for model validation. 


\documentclass[11pt]{article}
\usepackage{durnew,vdmsl-2e}

% Page usage. Big A4 size.
\oddsidemargin 1pt
\evensidemargin 1pt
\marginparwidth 30pt
\topmargin 1pt
% \headheight 12pt
% \headsep 8pt
\footskip 24pt
\textheight 650pt
\textwidth 460pt
% End of page usage.


\itletters

\parindent=0pt
\parskip=.5\baselineskip


\title{Modelling Railway Interlocking Systems\thanks{This work is supported by the
    Danish State Railways, by the project Mathematical
    Modelling of Computer Based Systems (MMaDS) at the Technical University of
    Denmark, and by the Commission of the European Communities (CEC),
    ESPRIT, Basic Research Action
    proj.\ no.\ 7071: ``ProCoS II: Provably Correct Systems''.}}
\author{Kirsten Mark Hansen\thanks{Department of Computer Science, Technical University
of Denmark Build. 344, DK-2800 Lyngby, Denmark, and
Danish State Railways, Infrastructure Management-Consult, Signalling
Safety Assessment, 
S\o lvgade 40, opg.\ F, 4.sal, DK-1349 K\o benhavn K, Denmark}}

\date{\today}

\begin{document}

\maketitle

\begin{abstract}
In this report we present a model of interlocking systems, and
describe how the model may be validated by simulation. Station
topologies are modelled by graphs in which the nodes denote track
segments, and the edges denote connectivity for train traffic. Points
and signals are modelled by annotations on the edges, thereby
restricting the driving possibilities. We define the safe station
states as predicates on the graph, and present a first step towards an
implementation of
these predicates. Both the model of station topologies, the safety
requirements, and the implementation are
validated using the VDM tool-box. The model development
illustrates how concepts may be captured and validated for a
non-trivial system.

\paragraph{CR Categories:} D.2.1, D.2.10, D.3.1, J.7
\paragraph{CR General Terms:} Design. 
\paragraph{Keywords:} Mathematical modelling, application of formal
methods, VDM-SL, stepwise
development, model validation, simulation, VDM tool-box, 
railway interlocking.
\end{abstract}


\section{Introduction}

The task of an
interlocking system is primarily to prevent trains from colliding, and
derailing, while at the same time allowing train movements. The Danish State
Railways
(DSB) has built and used interlocking systems for the past 150 years. The first
interlocking systems were pure mechanical systems, but as electricity became
common, the systems have developed through
electro mechanical to relay based systems and more recently to
computer based systems. Interlocking systems are safety
critical systems, so there is a need to ensure that the system prevents
dangerous
situations. Therefore DSB recently
decided to investigate the possibilities of using
formal software development for interlocking systems. This report presents
some of the
results of this work, which has been done as a cooperation between DSB and
the Department of Computer Science at the Technical University of
Denmark. Parts of this report have previously been published
in \cite{nsdcs94}, and in  \cite{fme94}.

In making a
formal model of interlocking systems,
we encountered a number of
difficulties, which seem to be general problems in
system modelling:

\begin{itemize}
\item no overall system requirements

\item a lack of precise concept definitions

\item the persons who develop
  interlocking systems had no knowledge of formal methods

\item we had no knowledge of interlocking systems.
\end{itemize}

The missing overall system requirements was a major problem.
There were informal specifications like
\cite{kravspec}, but they only contained low
level requirements like: `if the switching of a
point is not terminated within 20 seconds, the setting of the train route is
interrupted'.  There were no requirements stating the
purpose
of an interlocking system, and it turned out that
defining the overall task of an interlocking system was not a simple task.
One requirement could for instance be that an interlocking system
should prevent train
collision, but this is a matter for discussion, as joining
two trains into one strictly speaking is a train collision.

The second point was a lack of precise concept definitions.
Some concepts were not defined at all, e.g.\ train collision, and for
other concepts, the definition has changed over time, e.g.\ naming of
point branches.

Having made a system model there are generally two equal important aspects one
would like to investigate:

\begin{enumerate}
\item Is the model an acceptable model of  reality?

\item How is the model implemented?
\end{enumerate}

In formal program development one tends to focus on
the second aspect, but due to the difficulties mentioned above, we decided
to take validation seriously. 
Having
investigated a number of validation techniques, e.g.\ interviews,
structured reading \cite{validate,king}, symbolic execution of
specifications \cite{kneuper}, and prototyping (simulation)
\cite{prototyping}, we decided primarily to use simulation.
There were several reasons for this decision. First simulation is a
well-established technique in conventional mathematical modelling.
Next we had used interviews to reveal the requirements and therefore,
we did not trust interviews to be the only validation technique.
Structured reading means checking that there are no missing
references in a document, that all functions used are defined etc.,
but in choosing an executable specification language, we get this for
free. Finally symbolic execution of specifications did not fit very
well to our application as we had a specification with a small state
space and only one operation. 
By writing the specifications in VDM-SL
\cite{vdmsl}, and by using the VDM tool-box \cite{toolbox}, we were
able to execute the specification and thereby validate it. 

We do not formalise a particular interlocking system, but focus on
the principles and concepts upon which such systems are built. We take
interlocking systems used by the
DSB as starting point. Interlocking systems from other
countries may differ from the systems described here.

Other models
for interlocking systems
have been made in Higher Order Logic \cite{wai,morley3}, in
CCS \cite{morley2}, using double node graphs \cite{markus2,markus1}, and by
modelling tracks as units with end points \cite{sorenprehn}.
Many researchers have furthermore used railway applications as case studies,
e.g.\
\cite{jus,broy}, but this work is not directly applicable to our work, 
as it  either presents toy examples, or examples which have nothing to
do with interlocking, like e.g. level crossings \cite{jus}.

The content of the report is as follows: In Section~\ref{station} we
develop a model of station topologies and define their 
state spaces. In
Section~\ref{invar}, we define the safety requirements which an
interlocking system should fulfill, and in Section~\ref{intpre} we
present an implementation of the requirements. We
conclude in Section~\ref{concl} by
discussing the model development and executable
specifications for model validation. 


\documentclass[11pt]{article}
\usepackage{durnew,vdmsl-2e}

% Page usage. Big A4 size.
\oddsidemargin 1pt
\evensidemargin 1pt
\marginparwidth 30pt
\topmargin 1pt
% \headheight 12pt
% \headsep 8pt
\footskip 24pt
\textheight 650pt
\textwidth 460pt
% End of page usage.


\itletters

\parindent=0pt
\parskip=.5\baselineskip


\title{Modelling Railway Interlocking Systems\thanks{This work is supported by the
    Danish State Railways, by the project Mathematical
    Modelling of Computer Based Systems (MMaDS) at the Technical University of
    Denmark, and by the Commission of the European Communities (CEC),
    ESPRIT, Basic Research Action
    proj.\ no.\ 7071: ``ProCoS II: Provably Correct Systems''.}}
\author{Kirsten Mark Hansen\thanks{Department of Computer Science, Technical University
of Denmark Build. 344, DK-2800 Lyngby, Denmark, and
Danish State Railways, Infrastructure Management-Consult, Signalling
Safety Assessment, 
S\o lvgade 40, opg.\ F, 4.sal, DK-1349 K\o benhavn K, Denmark}}

\date{\today}

\begin{document}

\maketitle

\begin{abstract}
In this report we present a model of interlocking systems, and
describe how the model may be validated by simulation. Station
topologies are modelled by graphs in which the nodes denote track
segments, and the edges denote connectivity for train traffic. Points
and signals are modelled by annotations on the edges, thereby
restricting the driving possibilities. We define the safe station
states as predicates on the graph, and present a first step towards an
implementation of
these predicates. Both the model of station topologies, the safety
requirements, and the implementation are
validated using the VDM tool-box. The model development
illustrates how concepts may be captured and validated for a
non-trivial system.

\paragraph{CR Categories:} D.2.1, D.2.10, D.3.1, J.7
\paragraph{CR General Terms:} Design. 
\paragraph{Keywords:} Mathematical modelling, application of formal
methods, VDM-SL, stepwise
development, model validation, simulation, VDM tool-box, 
railway interlocking.
\end{abstract}


\section{Introduction}

The task of an
interlocking system is primarily to prevent trains from colliding, and
derailing, while at the same time allowing train movements. The Danish State
Railways
(DSB) has built and used interlocking systems for the past 150 years. The first
interlocking systems were pure mechanical systems, but as electricity became
common, the systems have developed through
electro mechanical to relay based systems and more recently to
computer based systems. Interlocking systems are safety
critical systems, so there is a need to ensure that the system prevents
dangerous
situations. Therefore DSB recently
decided to investigate the possibilities of using
formal software development for interlocking systems. This report presents
some of the
results of this work, which has been done as a cooperation between DSB and
the Department of Computer Science at the Technical University of
Denmark. Parts of this report have previously been published
in \cite{nsdcs94}, and in  \cite{fme94}.

In making a
formal model of interlocking systems,
we encountered a number of
difficulties, which seem to be general problems in
system modelling:

\begin{itemize}
\item no overall system requirements

\item a lack of precise concept definitions

\item the persons who develop
  interlocking systems had no knowledge of formal methods

\item we had no knowledge of interlocking systems.
\end{itemize}

The missing overall system requirements was a major problem.
There were informal specifications like
\cite{kravspec}, but they only contained low
level requirements like: `if the switching of a
point is not terminated within 20 seconds, the setting of the train route is
interrupted'.  There were no requirements stating the
purpose
of an interlocking system, and it turned out that
defining the overall task of an interlocking system was not a simple task.
One requirement could for instance be that an interlocking system
should prevent train
collision, but this is a matter for discussion, as joining
two trains into one strictly speaking is a train collision.

The second point was a lack of precise concept definitions.
Some concepts were not defined at all, e.g.\ train collision, and for
other concepts, the definition has changed over time, e.g.\ naming of
point branches.

Having made a system model there are generally two equal important aspects one
would like to investigate:

\begin{enumerate}
\item Is the model an acceptable model of  reality?

\item How is the model implemented?
\end{enumerate}

In formal program development one tends to focus on
the second aspect, but due to the difficulties mentioned above, we decided
to take validation seriously. 
Having
investigated a number of validation techniques, e.g.\ interviews,
structured reading \cite{validate,king}, symbolic execution of
specifications \cite{kneuper}, and prototyping (simulation)
\cite{prototyping}, we decided primarily to use simulation.
There were several reasons for this decision. First simulation is a
well-established technique in conventional mathematical modelling.
Next we had used interviews to reveal the requirements and therefore,
we did not trust interviews to be the only validation technique.
Structured reading means checking that there are no missing
references in a document, that all functions used are defined etc.,
but in choosing an executable specification language, we get this for
free. Finally symbolic execution of specifications did not fit very
well to our application as we had a specification with a small state
space and only one operation. 
By writing the specifications in VDM-SL
\cite{vdmsl}, and by using the VDM tool-box \cite{toolbox}, we were
able to execute the specification and thereby validate it. 

We do not formalise a particular interlocking system, but focus on
the principles and concepts upon which such systems are built. We take
interlocking systems used by the
DSB as starting point. Interlocking systems from other
countries may differ from the systems described here.

Other models
for interlocking systems
have been made in Higher Order Logic \cite{wai,morley3}, in
CCS \cite{morley2}, using double node graphs \cite{markus2,markus1}, and by
modelling tracks as units with end points \cite{sorenprehn}.
Many researchers have furthermore used railway applications as case studies,
e.g.\
\cite{jus,broy}, but this work is not directly applicable to our work, 
as it  either presents toy examples, or examples which have nothing to
do with interlocking, like e.g. level crossings \cite{jus}.

The content of the report is as follows: In Section~\ref{station} we
develop a model of station topologies and define their 
state spaces. In
Section~\ref{invar}, we define the safety requirements which an
interlocking system should fulfill, and in Section~\ref{intpre} we
present an implementation of the requirements. We
conclude in Section~\ref{concl} by
discussing the model development and executable
specifications for model validation. 


\documentclass[11pt]{article}
\usepackage{durnew,vdmsl-2e}

% Page usage. Big A4 size.
\oddsidemargin 1pt
\evensidemargin 1pt
\marginparwidth 30pt
\topmargin 1pt
% \headheight 12pt
% \headsep 8pt
\footskip 24pt
\textheight 650pt
\textwidth 460pt
% End of page usage.


\itletters

\parindent=0pt
\parskip=.5\baselineskip


\title{Modelling Railway Interlocking Systems\thanks{This work is supported by the
    Danish State Railways, by the project Mathematical
    Modelling of Computer Based Systems (MMaDS) at the Technical University of
    Denmark, and by the Commission of the European Communities (CEC),
    ESPRIT, Basic Research Action
    proj.\ no.\ 7071: ``ProCoS II: Provably Correct Systems''.}}
\author{Kirsten Mark Hansen\thanks{Department of Computer Science, Technical University
of Denmark Build. 344, DK-2800 Lyngby, Denmark, and
Danish State Railways, Infrastructure Management-Consult, Signalling
Safety Assessment, 
S\o lvgade 40, opg.\ F, 4.sal, DK-1349 K\o benhavn K, Denmark}}

\date{\today}

\begin{document}

\maketitle

\begin{abstract}
In this report we present a model of interlocking systems, and
describe how the model may be validated by simulation. Station
topologies are modelled by graphs in which the nodes denote track
segments, and the edges denote connectivity for train traffic. Points
and signals are modelled by annotations on the edges, thereby
restricting the driving possibilities. We define the safe station
states as predicates on the graph, and present a first step towards an
implementation of
these predicates. Both the model of station topologies, the safety
requirements, and the implementation are
validated using the VDM tool-box. The model development
illustrates how concepts may be captured and validated for a
non-trivial system.

\paragraph{CR Categories:} D.2.1, D.2.10, D.3.1, J.7
\paragraph{CR General Terms:} Design. 
\paragraph{Keywords:} Mathematical modelling, application of formal
methods, VDM-SL, stepwise
development, model validation, simulation, VDM tool-box, 
railway interlocking.
\end{abstract}


\section{Introduction}

The task of an
interlocking system is primarily to prevent trains from colliding, and
derailing, while at the same time allowing train movements. The Danish State
Railways
(DSB) has built and used interlocking systems for the past 150 years. The first
interlocking systems were pure mechanical systems, but as electricity became
common, the systems have developed through
electro mechanical to relay based systems and more recently to
computer based systems. Interlocking systems are safety
critical systems, so there is a need to ensure that the system prevents
dangerous
situations. Therefore DSB recently
decided to investigate the possibilities of using
formal software development for interlocking systems. This report presents
some of the
results of this work, which has been done as a cooperation between DSB and
the Department of Computer Science at the Technical University of
Denmark. Parts of this report have previously been published
in \cite{nsdcs94}, and in  \cite{fme94}.

In making a
formal model of interlocking systems,
we encountered a number of
difficulties, which seem to be general problems in
system modelling:

\begin{itemize}
\item no overall system requirements

\item a lack of precise concept definitions

\item the persons who develop
  interlocking systems had no knowledge of formal methods

\item we had no knowledge of interlocking systems.
\end{itemize}

The missing overall system requirements was a major problem.
There were informal specifications like
\cite{kravspec}, but they only contained low
level requirements like: `if the switching of a
point is not terminated within 20 seconds, the setting of the train route is
interrupted'.  There were no requirements stating the
purpose
of an interlocking system, and it turned out that
defining the overall task of an interlocking system was not a simple task.
One requirement could for instance be that an interlocking system
should prevent train
collision, but this is a matter for discussion, as joining
two trains into one strictly speaking is a train collision.

The second point was a lack of precise concept definitions.
Some concepts were not defined at all, e.g.\ train collision, and for
other concepts, the definition has changed over time, e.g.\ naming of
point branches.

Having made a system model there are generally two equal important aspects one
would like to investigate:

\begin{enumerate}
\item Is the model an acceptable model of  reality?

\item How is the model implemented?
\end{enumerate}

In formal program development one tends to focus on
the second aspect, but due to the difficulties mentioned above, we decided
to take validation seriously. 
Having
investigated a number of validation techniques, e.g.\ interviews,
structured reading \cite{validate,king}, symbolic execution of
specifications \cite{kneuper}, and prototyping (simulation)
\cite{prototyping}, we decided primarily to use simulation.
There were several reasons for this decision. First simulation is a
well-established technique in conventional mathematical modelling.
Next we had used interviews to reveal the requirements and therefore,
we did not trust interviews to be the only validation technique.
Structured reading means checking that there are no missing
references in a document, that all functions used are defined etc.,
but in choosing an executable specification language, we get this for
free. Finally symbolic execution of specifications did not fit very
well to our application as we had a specification with a small state
space and only one operation. 
By writing the specifications in VDM-SL
\cite{vdmsl}, and by using the VDM tool-box \cite{toolbox}, we were
able to execute the specification and thereby validate it. 

We do not formalise a particular interlocking system, but focus on
the principles and concepts upon which such systems are built. We take
interlocking systems used by the
DSB as starting point. Interlocking systems from other
countries may differ from the systems described here.

Other models
for interlocking systems
have been made in Higher Order Logic \cite{wai,morley3}, in
CCS \cite{morley2}, using double node graphs \cite{markus2,markus1}, and by
modelling tracks as units with end points \cite{sorenprehn}.
Many researchers have furthermore used railway applications as case studies,
e.g.\
\cite{jus,broy}, but this work is not directly applicable to our work, 
as it  either presents toy examples, or examples which have nothing to
do with interlocking, like e.g. level crossings \cite{jus}.

The content of the report is as follows: In Section~\ref{station} we
develop a model of station topologies and define their 
state spaces. In
Section~\ref{invar}, we define the safety requirements which an
interlocking system should fulfill, and in Section~\ref{intpre} we
present an implementation of the requirements. We
conclude in Section~\ref{concl} by
discussing the model development and executable
specifications for model validation. 


\include{rail.vdmsl}

\end{document}


\end{document}


\end{document}


\end{document}
