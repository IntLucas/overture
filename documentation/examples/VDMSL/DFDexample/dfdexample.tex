\documentclass[11pt]{article}
\usepackage{a4}
\usepackage{makeidx}
%\usepackage{vdmsl-2e}
%\usepackage{color}
\usepackage{overture}
\usepackage{termref}
\usepackage{epsf}
\usepackage{latexsym}
\usepackage{longtable}

\newcommand{\StateDef}[1]{{\bf #1}}
\newcommand{\TypeDef}[1]{{\bf #1}}
\newcommand{\TypeOcc}[1]{{\it #1}}
\newcommand{\FuncDef}[1]{{\bf #1}}
\newcommand{\FuncOcc}[1]{#1}
\newcommand{\ModDef}[1]{{\tiny #1}}

\makeindex
%\documentstyle[epsf,a4,11pt,vdmsl_v1.1.31,makeidx,termref]{article}
\newcommand{\figdir}{/home/peter/toolbox}
\newcommand{\SA}{{\small SA}}
\newcommand{\SD}{{\small SD}}
\newcommand{\SASD}{{\small SA/SD}}
\newcommand{\VDM}{{\small VDM}}
\newcommand{\VDMSL}{{\small VDM-SL}}
\newcommand{\DFD}{{\small DFD}}
\newcommand{\DFDs}{{\small DFD}s}

\newcommand{\makefigure}[3]{\begin{figure}[ht]
{\leavevmode
\centering
\epsfbox{\figdir/#1.eps}
\caption{#2}\label{#3}}
\end{figure}}

\newcommand{\documenttype}{document}
\makeindex
\begin{document}

\title{A Formal Semantics of Data Flow Diagrams}
\author{Peter Gorm Larsen, Nico Plat and Hans Toetenel}
\date{November 1993}
\maketitle
\begin{abstract}
This document presents a full version of the
formal semantics of data flow diagrams reported in \cite{Larsen&93b}.
Data Flow Diagrams are used in Structured Analysis and are
 based on an abstract model for data flow transformations.
The semantics consists of a collection of \VDM\ functions,
transforming an abstract syntax representation of a data flow
diagram into an abstract syntax representation of a \VDM\
specification. Since this transformation is executable, it
becomes possible to provide a software analyst/designer with two `views' of
the system being modeled: a graphical view in terms of a data
flow diagram, and a textual view in terms of a \VDM\
specification. The specification presented in this document have been
processed by The IFAD VDM-SL Toolbox \cite{Lassen93} and the \LaTeX\ output is
produced directly by means of this tool. The complete transformation has
been syntax-checked, type-checked and tested using the
{\small IFAD VDM-SL}\ Toolbox \cite{Lassen93}; this has given us
confidence that the transformation we have defined is a reasonable one.
\end{abstract}
\newpage
\tableofcontents
\newpage
\section{Introduction}

The introduction of formal methods in industrial organizations
may become easier if these methods can be used alongside the
more widely used conventional techniques for software development,
such as `structured methods'.
Structured methods are methods for software analysis and design,
based on the use of heuristics for making analysis and design decisions.
They provide a relatively well-defined path, often in a cookbook-like
fashion (hence the term `structured' methods), starting from the analysis
of software requirements and ending at system coding.
The design notations used are usually graphical and have no formal basis.
In that sense structured and formal methods can be regarded as complementary.
It is often suggested that the informal graphical notations as provided by
structured methods are intuitively appealing to software analysts/designers.
Therefore, a combined structured/formal method may not only increase the
understanding of the use of formal methods in the software process, but 
also may increase the acceptability of formal methods to these people.

Our work in this area so far has concentrated on combining {\em Structured
Analysis (SA)} \cite{Yourdon75,Demarco78,Gane77}
with the {\em Vienna Development Method (VDM)}
\cite{Bjorner&82b,Jones90a}; we provide a brief introduction to
\SA, but we refer to text books such as \cite{Jones90a} and
\cite{Andrews&91} for an introduction to \VDM.
We think that a well-integrated combination of notations can be achieved
by using {\em data flow diagrams (DFDs)} -- which we consider to be the main
design notation of \SA\ --
as a graphical view of the system and \VDM\ as a textual
view. These different views emphasize different aspects of the
specified system: the \DFD\ graphical view
focuses on an overview of the {\em structure} of the system, whereas
the \VDM\ textual view focuses on the detailed {\em functionality} of the system.
The base of a combined structured/formal method consists of a formally defined
relation between the structured method and the formal method.
In \cite{Plat&91a} we describe several approaches to modeling \DFDs\ 
using the \VDMSL\ specification language \cite{BSIVDM92b,Dawes91}.
In this paper we discuss one such particular model in more detail,
thus essentially providing a `formal semantics' of \DFDs.
A discussion on the methodological aspects of the approach can be found in
\cite{Larsen&91b}.

The remainder of this paper is organized as follows.
In the following section a brief introduction to \SA\ is given, focusing on the
use of \DFDs.
In Section~\ref{sec:motivation} we describe our strategy for
transforming \DFDs\ into \VDM\ specifications, paying attention to the
limitations of our approach.
The main part of this paper is Section~\ref{transsec}, in which the formal
transformation from \DFDs\ to \VDM\ is presented.
Emphasis is put on the motivation for the choices
made in the transformation. The main aspects of the transformation
itself are described using \VDM\ functions together
with a number of examples.
In Section~\ref{sec:conclusions} we give an overview of related
work on formal semantics for \DFDs, and present some conclusions and
ideas for future work in this area.
This is followed by a list of references, and two appendices and an
index for the complete specification.
In the first appendix, we describe the abstract syntax representation
of \DFDs. In the second appendix we present the abstract syntax 
representation we use for the generated
\VDM\ specifications. This is a subset of the one used
in \cite{BSIVDM92b}\footnote{%
To be precise, the abstract syntax used for \VDM\ specifications is
a partof the one
called the `Outer Abstract Syntax' in \cite{BSIVDM92b}; a lack of knowledge about
this Outer Abstract Syntax
does not affect the understanding of this document, however.}).

\section{Overview of Structured Analysis}
\label{sec:sa}

{\em Structured Analysis ({\small\it SA})}
\cite{Yourdon75,Demarco78,Gane77} is one of
the most widely used methods for software analysis.
Often it is used in combination with {\em Structured Design ({\small\it SD})}
\cite{Yourdon75}; the resulting combination is called {\em \SASD}.
The approach to analysis taken in \SA\ is to concentrate on the functions
to be carried out by the system, using data flow abstraction to
describe the flow of data through a network of transforming
processes, called data transformers, together with access to data stores.
Such a network, which is the most important design product of \SA,
is called a {\em data flow diagram (DFD)}. The original version of
\SA\ was meant to be used to model sequential systems.  
A \DFD\ is a {\em directed graph} consisting of elementary building blocks.
Each building block has a graphical notation (figure~\ref{figexternalprocess}).

%\makefigure{building_blocks}{Elementary building blocks of a DFD}{figexternalprocess}

Through the years several dialects have evolved and extensions
have been defined (e.g. {\small SSADM}\ \cite{Longworth&86} and {\small SA/RT}\
\cite{Ward85}), 
but we limit ourselves to \DFDs\ with a sequential model and
a small number of building blocks:
\begin{itemize}
\item
{\em Data transformers}. Data transformers denote a transformation from
(an arbitrary number of) input values to
(an arbitrary number of) output values, possibly with side effects.
\item
{\em Data flows}. Data flows are represented as arrows, connecting one
data transformer to another. They represent a flow of data between the
data transformers they connect.
The flow of data is unidirectional in the direction of the arrow.
\item
{\em Data stores}. Data stores provide for (temporary) storage of data.
\item
{\em External processes.}
External processes are processes that are not part of the system but belong
to the outside world. They are used to show where the input to the system
is coming from and where the output of the system is going to.
\end{itemize}

\DFD s are used to model the {\em information flow} through a system.
As such they provide a limited view of the system:
in their most rudimentary form they neither show the control flow of the system
nor any timing aspects.
Therefore, \DFDs\ are often combined with data dictionaries,
control flow diagrams, state transition diagrams, decision tables
and mini-specifications to provide a comprehensive view of all the 
aspects of the system.

The process of constructing a \DFD\ is an iterative process.
Initially, the system to be designed is envisaged as one large data transformer,
getting input from and providing output to external processes.
This initial, high-level \DFD\ is called a {\em context diagram}.
The next step is the {\em decomposition}
of the context diagram into a network of data transformers,
the total network providing the same functionality as the original
context diagram.
This process is repeated for each data transformer until the analyst/designer
considers all the data transformers in the \DFD\ to be primitive,
i.e. each data transformer performs a simple operation that does not
need to be further decomposed.
We call such a collection of \DFDs, describing the same system but at different
levels of abstraction, a {\em hierarchy of DFDs}.

\section{Approach to the Transformation}
\label{sec:motivation}

Before presenting the formal transformation from \DFDs\ to \VDM\ we first explain
the underlying strategy for the transformation and the limitations
imposed upon the \DFDs\ to make our transformation valid.

\subsection{Underlying strategy}

The starting point for our transformation is the work presented in
\cite{Plat&91a}, in which the general properties of two transformations from
DFDs to \VDM\ constructs are discussed.
The main difference between these two transformations is the way data flows are
modeled: in the first transformation they are modeled as (infinitely large)
queues, in the second transformation they are modeled as operations combining
the two data transformers connected by the data flow.
The advantage of the latter transformation is that a more abstract
interpretation of \DFDs\ can be achieved, because the transformation
solely focuses on modeling the information flow through a \DFD.
This is also the reason for choosing this transformation 
as the basis for the transformation described in this paper.
One simplification with respect to the transformation described in
\cite{Plat&91a} is that the latter is
more general because the order
in which the `underlying' operations are called is left unspecified
(i.e. it is loosely specified),
which makes the operation modeling the data flow rather complicated.
In this paper, however, we are dealing with purely sequential systems,
and therefore we can assume that data flows between two
data transformers are `direct' in the sense that 
the data transformer that uses the data flow as input cannot be
called before the data transformer that uses the data flow as output.

\subsection{Transformation of DFD building blocks}


When providing a formal semantics for \DFDs\ it is
important to decide whether the \DFD\ is intended to model
a concurrent system or a sequential system. More recent versions
of \SA\ (like {\em SA/RT} \cite{Ward85}) include concurrency and can be
used to develop real-time systems. However, originally \SA\
was intended for the development of information systems
implemented in traditional imperative programming languages.
In that situation it seems natural to interpret the data transformers
as {\em functions} or
{\em operations} which, given input data, sequentially perform computations and
produce output data.
If the data flow diagrams are used to model a concurrent system
it is more natural to interpret data transformers as {\em processes},
possibly executing in parallel.
Since we restrict ourselves to sequential systems we model data transformers
as {\em VDM operations}\footnote{Data transformers neither having access to
data stores nor being connected to external processes can also be modeled
as {\em VDM functions}. In our approach only {\tiny VDM}
operations are used because
we want each different type of construct in a {\tiny DFD} to be mapped to 
(semantically) the same construct in {\tiny VDM}.
{\tiny VDM} functions and {\tiny VDM} operations
(without side-effects) semantically differ in the way {\em looseness} is interpreted
(see \cite{IFIP}).}.

To ensure that the structure of the \VDM\ specification
resembles the structure of the \DFD,
we group the operations modeling data transformers at the same level in
a hierarchy of \DFDs\ together in `modules'\footnote{%
{\tiny VDM-SL} as described in \cite{BSIVDM92b} has no structuring mechanism.
The structuring mechanism we used is based on a proposal by Bear
\cite{Bear88}. The constructs we use are simple so that an intuitive
interpretation suffices.}
importing the necessary types and
operations needed for the data transformers
(figure~\ref{hierarchy}).

%\makefigure{hierarchy}{Transformation of a hierarchy of DFDs into a VDM module structure (example)}{hierarchy}


{\em External processes} can be considered as processes `executing' in parallel with
the specified system.
In our approach we model the data flows from and to external processes
as {\em state components} in the \VDM\ specification.
This is a minor difference with the transformation presented in
\cite{Plat&91a},
in which external processes are regarded as part of the system and
are therefore modeled as \VDM\ operations in the same way as data transformers.


{\em Data stores} are modeled as \VDM\ {\em state components}.
This corresponds to the fact that data transformers
(which can be used to access and change data stores)
are modeled as \VDM\ operations,
the constructs in \VDMSL\ that can access and change state components.


We envisage {\em data flows} as constructs which can combine
two data transformers by providing communication facilities between
these two data transformers.
A data flow is, therefore, modeled as an operation
calling the operations that model the two data transformers connected by
the data flow. 
In this way a process of combining data transformers can be started during
which in each step two data transformers (connected by a data flow) are
integrated into a higher level data transformer, finally resulting in
the context diagram.

Generalizing this approach, we have chosen to combine 
all the data transformers in a \DFD\ into a
higher level data transformer in {\em one} step.
The data transformer constructed in this way is modeled as a \VDM\ operation.

\subsection{Limitations imposed upon the DFDs}

Besides restricting the expressibility of the kind of \DFDs\
for which we are able to provide semantics to sequential systems, we
assume that:

\begin{itemize}
\item Data flows not connected
      to an external process must form an acyclic
      graph at each level in the hierarchy of \DFDs. This
      is necessary because in our transformation we provide both
	  explicit \VDM\ specifications as well as implicit \VDM\ specifications
	  as models for \DFDs. Allowing general cyclic \DFDs\ would make the 
	  transformation into an explicit \VDM\ specification impossible.
	  The restriction furthermore simplifies the
      transformation of \DFDs\ into implicit \VDM\ specifications.
      In Section~\ref{depend} we come back to this restriction in more
	  detail.
\item There is a one-to-one mapping between the
      input to the system and the output from the system.
	  One-to-many mappings and many-to-one mappings are a common
	  problem when interpreting \DFDs, described in more detail
	  in \cite{Alabiso88}\footnote{%
      In \cite{Alabiso88} this problem is called {\em I/O uncohesiveness}.
      I/O uncohesiveness occurs if either a data transformer
      must consume several pieces of input data before generating
      output data, or if a data transformer generates pieces of output
      independently of all other inputs and outputs.
      Alabiso describes a solution called `the burial method',
	  centered around the generation of terminator symbols
      which indicate that `something is missing'.}.
	  However, we are not entirely satisfied with the solution proposed by
	  Alabiso, and since in our experience most of the \DFDs\ with
	  one-to-many or many-to-one mappings should be regarded as design
	  products and not as specification products, we feel that a
	  restriction to one-to-one mappings is not a serious one for our
	  purpose.
      Alternatively, the analyst may supply a mini-specification for 
      each non-primitive data transformer not obeying the restriction
	  of a one-to-one mapping between input and output.

\item To simplify the formal description the
data flows must have unique names at each level in the hierarchy
of \DFDs.
\end{itemize}


functions
  
  TransHDFD : HDFD * MSs * (<EXPL>|<IMPL>) ->  set of Module
  TransHDFD(hdfd,mss,style) ==
  let mainmod=MakeDFDModule(hdfd,mss,style) in
  let mk_(-,-,-,dfdmap,-)=hdfd in
  let mods= dunion {TransHDFD(dfd,mss,style)|dfd in set  rng dfdmap} in
  {mainmod} union mods;

  MakeDFDModule : HDFD * MSs * (<EXPL>|<IMPL>) -> Module
  MakeDFDModule(mk_(dfdid,dss,dfdtopo,dfdmap,dfdsig),mss,style) ==
  let int' = MakeInterface(dfdid,dss,dfdtopo,dfdsig,dfdmap),
      defs = MakeDefinitions(dfdid,dss,dfdtopo,dfdsig,mss,style) in
  mk_(ModIdConf(dfdid),int',defs);

  MakeInterface : DFDId * DSs * DFDTopo * DFDSig * DFDMap -> Interface
  MakeInterface(dfdid,dss,dfdtopo,dfdsig,dfdmap) ==
  let tmimp = MakeTypeModImp(dss, dom dfdtopo),
      dfdmimps = MakeDFDModImps(dom dfdmap,dfdsig),
      exp = MakeOpExp(dfdid,dfdsig(dfdid)) in
       mk_(({tmimp} union dfdmimps),exp)
  pre dfdid in set  dom dfdsig;
  
  MakeTypeModImp : DSs *  set of FlowId -> Import
  MakeTypeModImp(dss,fids) ==
  let tysigs={mk_TypeSig(DSIdConf(dsid))|dsid in set dss} union 
             {mk_TypeSig(FlowIdTypeConf(fid))|fid in set fids} in
  mk_(TypeModConf(),tysigs);
  
  MakeDFDModImps :  set of DFDId * DFDSig -> set of Import
  MakeDFDModImps(dfdids,dfdsig) ==
  {mk_(ModIdConf(id),{MakeOpSig(id,dfdsig(id))}) | id in set dfdids}
  pre dfdids subset  dom dfdsig;
  
  MakeOpExp : DFDId * Signature -> Export
  MakeOpExp(dfdid,sig) ==
  {MakeOpSig(dfdid,sig)};
  
  MakeOpSig : DFDId * Signature -> OpSig
  MakeOpSig(dfdid,sig) ==
  let opty = MakeOpType(sig),
      opst = MakeOpState(sig) in
  mk_OpSig(OpIdConf(dfdid),opty,opst);
  
  MakeOpType : Signature -> OpType
  MakeOpType(mk_(il,ol,-)) ==
  mk_OpType(MakeType(il),MakeType(ol));

  MakeType :  seq of FlowId -> [Type]
  MakeType(fidl) ==
  cases  len fidl:
    0 -> nil ,
    1 -> FlowIdTypeConf( hd fidl),
  others -> mk_ProductType([FlowIdTypeConf(fidl(i))|i in set inds fidl])
  end;
  
  MakeOpState : Signature ->  seq of Id
  MakeOpState(mk_(-,-,sl)) ==
  [let mk_(s,-)=sl(i) in
  StateVarConf(s)|i in set inds sl];

  MakeDefinitions : DFDId * DSs * DFDTopo * DFDSig * MSs * (<EXPL>|<IMPL>) ->
                    Definitions
  MakeDefinitions(dfdid,dss,dfdtopo,dfdsig,mss,style) ==
  let $st = MakeState(dfdid,dss,CollectExtDFs(dfdtopo)),
      msdescs = MakeMSDescs(dfdsig,mss),
      dfdop = MakeDFDOp(dfdid,dfdtopo,dfdsig,style) in
   if $st=nil 
   then {dfdop} union msdescs
   else {$st,dfdop} union msdescs;

  MakeState : DFDId * DSs *  set of FlowId -> [StateDef]
  MakeState(dfdid,dss,fids) ==
   if dss={} and fids={}
  then nil 
  else let fl=MakeFieldList(dss union fids) in
  mk_StateDef(StateIdConf(dfdid),fl)
  ;
  
  MakeFieldList :  set of StId ->  seq of Field
  MakeFieldList(ids) ==
   if ids={}
  then []
  else let id in set ids in
  [MakeField(id)]^MakeFieldList(ids \ {id})
  ;
  
  MakeField : StId -> Field
  MakeField(id) ==
  mk_Field(StateVarConf(id),StateTypeConf(id));

  MakeMSDescs : DFDSig * MSs ->  set of Definition
  MakeMSDescs(dfdsig,mss) ==
   if forall id in set  dom dfdsig&
   is_DFDId(id)
  then {}
  else let id in set  dom dfdsig be st is_MSId(id) in
  let def'= if id in set  dom mss
            then mss(id)
            else MakeOp(id,dfdsig(id))
   in
  {def'} union MakeMSDescs({id} <-: dfdsig,mss)
  ;
  
  MakeOp : MSId *  (seq of FlowId *  seq of FlowId * State) -> ImplOp
  MakeOp(msid,mk_($in,out,$st)) ==
  let partpl = MakeInpPar($in),
      residtp = MakeOutPair(out),
      $ext = MakeExt($st),
      body = mk_ImplOpBody(nil , mk_BoolLit(true)) in
  mk_ImplOp(OpIdConf(msid),partpl,residtp,$ext,body);
  
  MakeInpPar :  seq of FlowId ->  seq of ParType
  MakeInpPar(fidl) ==
  [mk_ParType(mk_PatternId(FlowIdVarConf(fidl(i))),FlowIdTypeConf(fidl(i)))
  | i in set inds fidl];

  MakeOutPair :  seq of FlowId -> [IdType]
  MakeOutPair(fidl) ==
  cases  len fidl:
    0 -> nil ,
    1 -> mk_IdType(FlowIdVarConf( hd fidl),FlowIdTypeConf( hd fidl)),
  others -> let t=mk_ProductType([FlowIdTypeConf(fidl(i))|i in set inds fidl]) 
            in
              mk_IdType(ResultIdConf(),t)
  end
  ;
  
  MakeExt : State ->  seq of ExtVarInf
  MakeExt($st) ==
  [MakeExtVar($st(i))|i in set inds $st];
  
  MakeExtVar : (StId * Mode) -> ExtVarInf
  MakeExtVar(mk_(id,mode)) ==
  mk_ExtVarInf(mode,VarConf(id),TypeConf(id));

  MakeDFDOp : DFDId * DFDTopo * DFDSig * (<EXPL>|<IMPL>) -> OpDef
  MakeDFDOp(dfdid,dfdtopo,dfdsig,style) ==
   if style=<EXPL>
  then MakeDFDExplOp(dfdid,dfdtopo,dfdsig)
  else MakeDFDImplOp(dfdid,dfdtopo,dfdsig)
  
   pre  if style=<EXPL>
        then pre_MakeDFDExplOp(dfdid,dfdtopo,dfdsig)
        else pre_MakeDFDImplOp(dfdid,dfdtopo,dfdsig);
  MakeDFDImplOp : DFDId * DFDTopo * DFDSig -> ImplOp
  MakeDFDImplOp(dfdid,dfdtopo,dfdsig) ==
  let mk_($in,out,$st)=dfdsig(dfdid) in
  let partpl = MakeInpPar($in),
      residtp = MakeOutPair(out),
      $ext = MakeExt($st),
      body = MakeImplOpBody(dfdid,dfdtopo,dfdsig) in
  mk_ImplOp(OpIdConf(dfdid),partpl,residtp,$ext,body)
  pre dfdid in set  dom dfdsig and 
      pre_MakeImplOpBody(dfdid,dfdtopo,dfdsig);

  MakeImplOpBody : DFDId * DFDTopo * DFDSig -> ImplOpBody
  MakeImplOpBody(dfdid,dfdtopo,dfdsig) ==
  let intm = {stid |-> 0|mk_(stid,-) in set CollectStIds( rng dfdsig)},
      maxm = {stid |-> Reduce(NoOfWr( rng dfdsig,stid))
             |mk_(stid, -) in set CollectStIds(rng dfdsig)} in
  let $pre  = MakePreExpr(dfdid,dfdtopo,dfdsig,intm,maxm),
      $post = MakePostExpr(dfdid,dfdtopo,dfdsig,intm,maxm) in
  mk_ImplOpBody($pre,$post)
  pre let intm = {stid |-> 0|mk_(stid,-) in set CollectStIds( rng dfdsig)},
          maxm = {stid |-> Reduce(NoOfWr( rng dfdsig,stid))
                 |mk_(stid,-) in set CollectStIds( rng dfdsig)} in
       pre_MakePreExpr(dfdid,dfdtopo,dfdsig,intm,maxm) and 
       pre_MakePostExpr(dfdid,dfdtopo,dfdsig,intm,maxm)

  types
  IntM = map StId to nat

  functions
  
  MakePreExpr : DFDId * DFDTopo * DFDSig * IntM * IntM -> Expr
  MakePreExpr(dfdid,dfdtopo,dfdsig,intm,maxm) ==
  let mk_(-,out,$st)=dfdsig(dfdid) in
  let fids = NeedsQuant(dfdtopo,dfdsig,{},{}),
  pred = MakePrePred(dfdtopo,dfdsig,intm,maxm) in
    if QuantNec(out,$st,fids,intm,maxm)
    then mk_ExistsExpr(MakeExistsBind(fids,$st,intm,maxm,<PRE>),pred)
    else pred
  pre dfdid in set  dom dfdsig;

  MakePrePred : DFDTopo * DFDSig * IntM * IntM -> Expr
  MakePrePred(dfdtopo,dfdsig,intm,maxm) ==
  let eos=ExecutionOrders(dfdtopo) in
  DBinOp(<OR>,{MakePreForEO(piseq,dfdsig,intm,maxm)|piseq in set eos});

  MakePreForEO :  seq1 of ProcId * DFDSig * IntM * IntM -> Expr
  MakePreForEO(piseq,dfdsig,intm,maxm) ==
  let nid= hd piseq in
  let intm'={stid |-> if mk_(stid, <READWRITE>) in set 
                         CollectStIds({dfdsig(nid)})
                      then intm(stid) + 1
                      else intm(stid)
            | stid in set  dom intm} in
  let $pre = MakeQuotedApply(nid,dfdsig(nid),intm',maxm,<PRE>,<PRE>),
      $post = MakeQuotedApply(nid,dfdsig(nid),intm',maxm,<PRE>,<POST>) in
   if  len piseq=1
   then $pre
   else let pred=mk_BinaryExpr($pre,<AND>,$post) in
     mk_BinaryExpr(pred,<AND>,MakePreForEO( tl piseq,dfdsig,intm',maxm));

  MakePostExpr : DFDId * DFDTopo * DFDSig * IntM * IntM -> Expr
  MakePostExpr(dfdid,dfdtopo,dfdsig,intm,maxm) ==
  let mk_(-,out,$st)=dfdsig(dfdid) in
  let fids = NeedsQuant(dfdtopo,dfdsig, elems out,{}) in
  let body = MakeInExpr(out,$st,fids,dfdtopo,dfdsig,intm,maxm) in
   if  len out<= 1
  then body
  else mk_LetExpr(MakePattern(out),ResultIdConf(),body)
  
  pre let mk_(-,out,$st)=dfdsig(dfdid) in
  let fids = NeedsQuant(dfdtopo,dfdsig, elems out,{}) in
  pre_MakeInExpr(out,$st,fids,dfdtopo,dfdsig,intm,maxm);

  MakeInExpr :  seq of FlowId * State *  set of FlowId * DFDTopo *
                DFDSig * IntM * IntM -> Expr
  MakeInExpr(out,$st,fids,dfdtopo,dfdsig,intm,maxm) ==
  let pred=MakePostPred(dfdtopo,dfdsig,intm,maxm) in
   if QuantNec(out,$st,fids,intm,maxm)
  then mk_ExistsExpr(MakeExistsBind(fids,$st,intm,maxm,<POST>),pred)
  else pred
  
  pre pre_MakeExistsBind(fids,$st,intm,maxm,<POST>);

  MakePostPred : DFDTopo * DFDSig * IntM * IntM -> Expr
  MakePostPred(dfdtopo,dfdsig,intm,maxm) ==
  let eos=ExecutionOrders(dfdtopo) in
  DBinOp(<OR>,{MakePostForEO(piseq,dfdsig,intm,maxm)|piseq in set eos});

  MakePostForEO :  seq1 of ProcId * DFDSig * IntM * IntM -> Expr
  MakePostForEO(piseq,dfdsig,intm,maxm) ==
  let nid= hd piseq in
  let intm'={stid |-> if mk_(stid, <READWRITE>) in set 
                         CollectStIds({dfdsig(nid)})
                      then intm(stid) + 1
                      else intm(stid)
            | stid in set  dom intm} in
  let $pre = MakeQuotedApply(nid,dfdsig(nid),intm',maxm,<POST>, <PRE>),
      $post = MakeQuotedApply(nid,dfdsig(nid),intm',maxm,<POST>,<POST>) in
   if  len piseq=1
  then mk_BinaryExpr($pre,<AND>,$post)
  else let pred=mk_BinaryExpr($pre,<AND>,$post) in
  mk_BinaryExpr(pred,<AND>,MakePostForEO( tl piseq,dfdsig,intm',maxm))
  
  pre let nid= hd piseq in
      nid in set  dom dfdsig and 
      pre_MakeQuotedApply(nid,dfdsig(nid),intm,maxm,<POST>,<PRE>) and 
      pre_MakeQuotedApply(nid,dfdsig(nid),intm,maxm,<POST>,<POST>);

  MakeExistsBind :  set of FlowId * State * IntM * IntM * (<PRE>|<POST>)
                    -> MultTypeBind
  MakeExistsBind(fs,$st,intm,maxm,c) ==
  let outl = MakeTypeBindList(fs),
  stl = [let mk_(s,-)=$st(i) in
         mk_TypeBind(MakePatternIds(s,intm(s)+1,maxm(s),c),
                     StateTypeConf(s))
        |i in set inds $st & let mk_(-,m)=$st(i) in m=<READWRITE>] in
  mk_MultTypeBind(outl^stl)
  pre forall mk_(s,<READWRITE>) in set  elems $st&
   s in set  dom intm and s in set  dom maxm;

  ExecutionOrders : DFDTopo ->  set of  seq1 of ProcId
  ExecutionOrders(dfdtopo) ==
  let top={mk_(fid,tid)|mk_(fid,tid) in set  rng dfdtopo &
            (is_DFDId(fid) or is_MSId(fid) or (fid = nil)) and 
            (is_DFDId(tid) or is_MSId(tid) or (tid = nil))},
      top2={mk_(fid,tid)|mk_(fid,tid) in set  rng dfdtopo &
            (is_DFDId(fid) or is_MSId(fid)) and 
            (is_DFDId(tid) or is_MSId(tid))} in
    let piset= dunion {{pi_1,pi_2}|mk_(pi_1,pi_2) in set top}\{nil} in
      {piseq | piseq in set PossibleSeqs(piset) &
               -- len piseq= card piset and  elems piseq=piset and 
               forall i,j in set inds piseq &
                  j<i => (piseq(j) not in set TransClosure(piseq(i),top2,{}))};

  MakeQuotedApply : (DFDId|MSId) * Signature * IntM * IntM * (<PRE>|<POST>) *
                    (<PRE>|<POST>) -> Apply
  MakeQuotedApply(id,mk_($in,out,$st),intm,maxm,c,c2) ==
  let inarg = [FlowIdVarConf($in(i))|i in set inds $in],
      oldstarg = [let mk_(s,m)=$st(i) in
                    if m=<READ>
                    then StateVarIntConf(s,intm(s),maxm(s),c)
                    else StateVarIntConf(s,intm(s) - 1,maxm(s),c)
                 |i in set inds $st],
      outarg = [FlowIdVarConf(out(i))|i in set inds out],
      starg = [let mk_(s,-)=$st(i) in
               StateVarIntConf(s,intm(s),maxm(s),c)
              |i in set inds $st & 
               let mk_(-,m)=$st(i) in m=<READWRITE>] in
   if c2=<PRE>
  then mk_Apply("pre_"^OpIdConf(id),inarg^oldstarg)
  else mk_Apply("post_"^OpIdConf(id),inarg^oldstarg^outarg^starg)
  
  pre forall mk_(s,m) in set  elems $st&
   s in set  dom intm and s in set  dom maxm and m=<READWRITE> => intm(s)>0;

  MakeDFDExplOp : DFDId * DFDTopo * DFDSig -> ExplOp
  MakeDFDExplOp(dfdid,dfdtopo,dfdsig) ==
  let mk_($in,-,-) = dfdsig(dfdid),
      eos = ExecutionOrders(dfdtopo),
      intm = {stid |-> 0|mk_(stid,-) in set CollectStIds( rng dfdsig)},
      maxm = {stid |-> Reduce(NoOfWr( rng dfdsig,stid))
             |mk_(stid,-) in set CollectStIds( rng dfdsig)} in
  let optype = MakeOpType(dfdsig(dfdid)),
      parms = [mk_PatternId(FlowIdVarConf($in(i)))|i in set inds $in],
      bodys = {MakeStmtForEO(piseq,dfdid,dfdsig)|piseq in set eos},
      $pre  = MakePreExpr(dfdid,dfdtopo,dfdsig,intm,maxm) in
  let body = MakeNonDetStmt(bodys) in
      mk_ExplOp(OpIdConf(dfdid),optype,parms,body,$pre)
  pre dfdid in set  dom dfdsig and 
      let intm = {stid |-> 0|mk_(stid,-) in set CollectStIds( rng dfdsig)},
          maxm = {stid |-> Reduce(NoOfWr( rng dfdsig,stid))
                 |mk_(stid,-) in set CollectStIds( rng dfdsig)} in
        pre_MakePreExpr(dfdid,dfdtopo,dfdsig,intm,maxm) and 
        forall piseq in set ExecutionOrders(dfdtopo)&
          pre_MakeStmtForEO(piseq,dfdid,dfdsig);

  MakeStmtForEO :  seq1 of ProcId * DFDId * DFDSig -> Stmt
  MakeStmtForEO(piseq,dfdid,dfdsig) ==
  let nid= hd piseq in
  let mk_(call,pat) = MakeCallAndPat(nid,dfdsig(nid)),
      kind = FindKind(dfdsig(nid)) in
   if  len piseq=1
   then let mk_(-,out,-)=dfdsig(dfdid) in 
         let return'=mk_Return(MakeResult(out)) in
          if kind=<OPRES>
          then mk_DefStmt(pat,call,return')
          else mk_Sequence([call,return'])
   else let rest=MakeStmtForEO( tl piseq,dfdid,dfdsig) in
         if kind=<OPRES>
         then mk_DefStmt(pat,call,rest)
         else if is_Sequence(rest)
              then let mk_Sequence(sl)=rest in
                    mk_Sequence([call]^sl)
              else mk_Sequence([call,rest])
  pre  hd piseq in set  dom dfdsig;
  
  MakeCallAndPat : (DFDId|MSId) * Signature -> Call * [Pattern]
  MakeCallAndPat(id,mk_($in,out,-)) ==
  let inarg = [FlowIdVarConf($in(i))|i in set inds $in],
      outarg = [FlowIdVarConf(out(i))|i in set inds out] in
  mk_(mk_Call(OpIdConf(id),inarg),MakePattern(outarg));
  
  FindKind : Signature -> <OPRES>|<OPCALL>
  FindKind(sig) ==
  cases sig:
    mk_(-,[],-) -> <OPCALL>,
    others -> <OPRES>
  end;
  
  MakePattern :  seq of Id -> [Pattern]
  MakePattern(idl) ==
  cases  len idl:
    0 -> nil ,
    1 -> mk_PatternId( hd idl),
  others -> mk_TuplePattern([mk_PatternId(idl(i)) | i in set inds idl])
  end;
  
  --MakeAssDef :  seq of Id ->  set of AssDef
  --MakeAssDef(idl) ==
  --{mk_AssDef(FlowIdVarConf(id),FlowIdTypeConf(id))|id in set  elems idl};
  
  MakeResult :  seq1 of Id -> Expr
  MakeResult(idl) ==
   if  len idl=1
  then FlowIdVarConf( hd idl)
  else mk_TupleConstructor([FlowIdVarConf(idl(i))|i in set inds idl])
  ;
  
  --SplitResult : Signature * Expr -> Expr
  --SplitResult(mk_(-,out,-),expr) ==
  -- if  len out<= 1
  --then expr
  --else let outarg=[FlowIdVarConf(out(i))|i in set inds out] in
  --mk_LetExpr(MakePattern(outarg),ResultIdConf(),expr)
  --;

  DBinOp : BinaryOp *  set of Expr -> Expr
  DBinOp(op,es) ==
  let e in set es in
   if  card es=1
  then e
  else mk_BinaryExpr(e,op,DBinOp(op, es \ {e}))
  
  pre es<>{};

  CollectExtDFs : DFDTopo ->  set of FlowId
  CollectExtDFs(dfdtopo) ==
  {fid|fid in set  dom dfdtopo & let mk_(pid_1,pid_2)=dfdtopo(fid) in
  is_EPId(pid_1) or is_EPId(pid_2)};
  
  NeedsQuant : DFDTopo * DFDSig *  set of FlowId *  set of ProcId ->  set of FlowId
  NeedsQuant(dfdtopo,dfdsig,notneeded,pids) ==
  let top={mk_(fid,tid)|mk_(fid,tid) in set  rng dfdtopo &
           (is_DFDId(fid) or is_MSId(fid)) and 
           (is_DFDId(tid) or is_MSId(tid))} in
   if  dom dfdsig=pids
   then {}
   else let pid in set  dom dfdsig \ pids in
        if TransClosure(pid,top,{})={} and EquivClass(top,{pid})= dom dfdsig
        then NeedsQuant(dfdtopo,dfdsig,notneeded,pids union {pid})
        else let mk_(-,out,-)=dfdsig(pid) in
              NeedsQuant(dfdtopo,dfdsig,notneeded,pids union {pid}) union 
              elems out \ notneeded;

  QuantNec :  seq of FlowId * State *  set of FlowId * IntM * IntM -> bool 
  QuantNec(out,$st,fids,intm,maxm) ==
  fids <> {} or
  -- (exists id in set  elems out&  id in set fids) or 
  (exists mk_(s,m) in set  elems $st& m=<READWRITE> and intm(s)<maxm(s))
  pre forall mk_(s,-) in set  elems $st&
   s in set  dom intm and s in set  dom maxm;
  
  MakeTypeBindList :  set of FlowId ->  seq of TypeBind
  MakeTypeBindList(fids) ==
   if fids={}
   then []
   else let fid in set fids in
         let first=mk_TypeBind([mk_PatternId(FlowIdVarConf(fid))],
                                             FlowIdTypeConf(fid)) in
           [first]^MakeTypeBindList(fids \ {fid});
  
  MakePatternIds: (Id | DSId) * nat  * nat  * (<PRE>|<POST>) -> seq of PatternId
  MakePatternIds(id, n, max, c) ==
  if (n = max) and (c = <POST>)
  then [mk_PatternId(StateVarConf(id))]
  else cases n:
       0      -> if c = <PRE>
                 then [mk_PatternId(StateVarConf(id))]
                 else [mk_PatternId(StateOldVarConf(id))],
       others -> MakePatternSeq(StateVarConf(id), n, max)
       end;
  
  MakePatternSeq: Id * nat * nat -> seq of PatternId
  MakePatternSeq(id, n, max) ==
  if n = max
  then [mk_PatternId(id ^ "'")]
  else [mk_PatternId(id ^ "'")] ^ MakePatternSeq(id ^ "'", n+1, max);

  EquivClass :  set of (ProcId * ProcId) *  set of (MSId|DFDId) ->
                set of (MSId|DFDId)
  EquivClass(top,ids) ==
   if exists mk_(fid,tid) in set top&
   (fid in set ids and tid not in set ids) or 
   (tid in set ids and fid not in set ids)
    then let mk_(fid,tid) in set top be st 
           (fid in set ids and tid not in set ids) or 
           (tid in set ids and fid not in set ids)
       in
         EquivClass(top,ids union {fid,tid})
  else ids;

  MakeNonDetStmt :  set of Stmt -> Stmt
  MakeNonDetStmt(stmts) ==
  cases  card stmts:
    1 -> let {s}=stmts in s,
  others -> mk_NonDetStmt(stmts)
  end
  
  pre  card stmts<>0;

  CollectStIds :  set of Signature ->  set of (StId * Mode)
  CollectStIds(sigs) ==
  -- let standms= dunion { elems $st|mk_(-,-,$st) in set sigs} in
  -- {stid|mk_(stid,-) in set standms};
  dunion { elems $st|mk_(-,-,$st) in set sigs};

  NoOfWr :  set of Signature * StId -> nat 
  NoOfWr(sigs,stid) ==
   if sigs={}
  then 0
  else let sig in set sigs in
  let mk_(-,-,$st)=sig in
   if mk_(stid,<READWRITE>) in set  elems $st
   then 1+NoOfWr(sigs \ {sig},stid)
   else NoOfWr(sigs \ {sig},stid);
  
  Reduce: nat -> nat
  Reduce(n) ==
  if (n = 0) or (n = 1)
  then n
  else n - 1;

  ModIdConf : DFDId -> Id
  ModIdConf(mk_DFDId(id)) ==
  id^"Module";
  
  StateIdConf : DFDId -> Id
  StateIdConf(mk_DFDId(id)) ==
  id^"State";
  
  DSIdConf : DSId -> Id
  DSIdConf(mk_DSId(id)) ==
  id;
  
  OpIdConf : MSId | DFDId | Id -> Id
  OpIdConf(id) ==
  cases id:
    mk_MSId(id'),
    mk_DFDId(id') -> id',
    others        -> id
  end;

  StateVarIntConf : (Id | DSId) * nat  * nat  * (<PRE>|<POST>) -> Id
  StateVarIntConf(id,n,max,c) ==
  if (max=n) and (c=<POST>)
  then StateVarConf(id)
  else cases n:
       0   ->  if c=<PRE>
               then StateVarConf(id)
               else StateOldVarConf(id),
       1    -> StateVarConf(id)^"'",
       others -> StateVarIntConf(id,n - 1,max,c)^"'"
       end;
  
  VarConf : StId -> Id
  VarConf(id) ==
  if is_DSId(id)
  then StateVarConf(id)
  else FlowIdVarConf(id);
  
  TypeConf : DSId|FlowId -> Id
  TypeConf(id) ==
  if is_DSId(id)
  then StateTypeConf(id)
  else FlowIdTypeConf(id);
  
  FlowIdVarConf : Id -> Id
  FlowIdVarConf(id) ==
  ToLower(id);
  
  FlowIdTypeConf : Id -> Id
  FlowIdTypeConf(id) ==
  ToUpper(id);
  
  StateTypeConf : Id | DSId -> Id
  StateTypeConf(id) ==
  ToUpper(id);
  
  StateVarConf : Id | DSId -> Id
  StateVarConf(id) ==
  ToLower(id);
  
  StateOldVarConf : Id | DSId -> Id
  StateOldVarConf(id) ==
  ToLower(id)^"old";
  
  TypeModConf : () -> Id
  TypeModConf() ==
  "TypeModule";
  
  ResultIdConf : () -> Id
  ResultIdConf() ==
  "r";
  
  PossibleSeqs: set of ProcId -> set of seq of ProcId
  PossibleSeqs(pids) ==
  if pids = {}
  then {}
  else if card pids = 1
       then {[pid]| pid in set pids}
       else let pid in set pids
            in
              let rest = PossibleSeqs(pids \ {pid})
              in
                dunion {InsertPId(pid, seq') | seq' in set rest};
  
  InsertPId: ProcId * seq of ProcId -> set of seq of ProcId
  InsertPId(pid, seq') ==
  {seq'(1,...,i) ^ [pid] ^ seq'(i+1,...,len(seq')) | i in set {0,...,len(seq')}};
  
  ToLower: Id | DSId | DFDId | EPId | MSId -> Id
  ToLower(id) ==
  let realid = cases id:
                 mk_DSId(id'),
                 mk_DFDId(id'),
                 mk_EPId(id'),
                 mk_MSId(id')  -> id',
                 others        -> id
               end
  in
    [LowerChar(realid(i)) | i in set inds realid];

  LowerChar: char -> char
  LowerChar(c) ==
  cases c:
  'A' -> 'a',
  'B' -> 'b',
  'C' -> 'c',
  'D' -> 'd',
  'E' -> 'e',
  'F' -> 'f',
  'G' -> 'g',
  'H' -> 'h',
  'I' -> 'i',
  'J' -> 'j',
  'K' -> 'k',
  'L' -> 'l',
  'M' -> 'm',
  'N' -> 'n',
  'O' -> 'o',
  'P' -> 'p',
  'Q' -> 'q',
  'R' -> 'r',
  'S' -> 's',
  'T' -> 't',
  'U' -> 'u',
  'V' -> 'v',
  'W' -> 'w',
  'X' -> 'x',
  'Y' -> 'y',
  'Z' -> 'z',
  others -> c
  end;
  
  
  ToUpper: Id | DSId | DFDId | EPId | MSId -> Id
  ToUpper(id) ==
  let realid = cases id:
                 mk_DSId(id'),
                 mk_DFDId(id'),
                 mk_EPId(id'),
                 mk_MSId(id')  -> id',
                 others        -> id
               end
  in
    [UpperChar(realid(i)) | i in set inds realid];
  
  UpperChar: char -> char
  UpperChar(c) ==
  cases c:
  'a' -> 'A',
  'b' -> 'B',
  'c' -> 'C',
  'd' -> 'D',
  'e' -> 'E',
  'f' -> 'F',
  'g' -> 'G',
  'h' -> 'H',
  'i' -> 'I',
  'j' -> 'J',
  'k' -> 'K',
  'l' -> 'L',
  'm' -> 'M',
  'n' -> 'N',
  'o' -> 'O',
  'p' -> 'P',
  'q' -> 'Q',
  'r' -> 'R',
  's' -> 'S',
  't' -> 'T',
  'u' -> 'U',
  'v' -> 'V',
  'w' -> 'W',
  'x' -> 'X',
  'y' -> 'Y',
  'z' -> 'Z',
  others -> c
  end

types
  
  SA = HDFD * DD * MSs
  inv mk_(hdfd,dd,-) == 
    FlowTypeDefined(hdfd,dd) and TopLevelSigOK(hdfd);

  HDFD = DFDId * DSs * DFDTopo * DFDMap * DFDSig;
--  inv mk_(id,dss,dfdtop,dfdmap,dfdsig) == 
--    DFDSigConsistent(id,dfdtop,dss,dfdmap,dfdsig) and 
--    LowerLevelUsed(dfdtop,dfdmap);
  
  DSs =  set of DSId;
  
  DSId :: seq of char;

  DFDTopo = map FlowId to ([ProcId] * [ProcId])
  inv dfdtopo == 
     let top={mk_(fid,tid)|mk_(fid,tid) in set rng dfdtopo &
                           (is_DFDId(fid) or is_MSId(fid)) and
                           (is_DFDId(tid) or is_MSId(tid))} in
       NotRecursive(top) and
     forall flowid in set  dom dfdtopo & FlowConnectOK(dfdtopo(flowid));
  
  FlowId = seq of char;
  
  ProcId = DFDId|MSId|EPId;
  
  DFDMap = map DFDId to HDFD;
  
  DFDSig = map (DFDId|MSId) to Signature;

  Signature = Input * Output * State
  inv mk_(-,out,sta) == 
    (sta=[]) => (out<>[]) and 
    (out=[]) => (exists mk_(-,m) in set elems sta & m=<READWRITE>);
  
  Input =  seq of FlowId;
  
  Output =  seq of FlowId;

  State =  seq of (StId * Mode);
  
  StId = DSId|FlowId;
  
  Mode = <READ>|<READWRITE>;
  
  DD = map Id to Type;
  
  MSs = map MSId to MS;
  
  MS = OpDef;
  
  DFDId :: seq of char;
  
  EPId :: seq of char;
  
  MSId :: seq of char
  
  -- InputId = seq of char;
  
  -- OutputId = seq of char;

functions
  
  FlowTypeDefined : HDFD * DD -> bool 
  FlowTypeDefined(mk_(-,-,dfdtop,-,-),dd) ==
    forall fid in set  dom dfdtop & FlowIdTypeConf(fid) in set  dom dd;

  TopLevelSigOK : HDFD -> bool 
  TopLevelSigOK(mk_(sysid,-,dfdtop,-,dfdsig)) ==
    sysid in set dom dfdsig and
    let mk_($in,out,$st)=dfdsig(sysid) in
    $in=[] and out=[] and
    forall flowid in set  dom dfdtop&
      let mk_(fid,tid)=dfdtop(flowid) in
        (is_EPId(fid) => mk_(flowid,<READ>) in set  elems $st) and 
        (is_EPId(tid) => mk_(flowid,<READWRITE>) in set  elems $st);

  DFDSigConsistent : DFDId * DFDTopo * DSs * DFDMap * DFDSig -> bool 
  DFDSigConsistent(id,dfdtop,dss,dfdmap,dfdsig) ==
    DSConnected(dss,dfdsig) and 
    SigsAllRight(dfdtop,dfdsig) and 
    IdsInSigsAvail(dss,dfdtop, rng dfdsig) and 
    SigsForAllUsedIds(id, rng dfdtop,dfdmap,dfdsig);
  
  DSConnected : DSs * DFDSig -> bool 
  DSConnected(dss,dfdsig) ==
  forall dsid in set dss&
   exists mk_(-,-,$st) in set  rng dfdsig&
   exists i in set inds $st&
   let mk_(id,-)=$st(i) in
    dsid=id;
  
  SigsAllRight : DFDTopo * DFDSig -> bool 
  SigsAllRight(dfdtop,dfdsig) ==
  forall flowid in set  dom dfdtop &
   cases dfdtop(flowid):
    mk_(id,mk_EPId(-)) -> let mk_(-,-,$st)=dfdsig(id) in
                            mk_(flowid,<READWRITE>) in set  elems $st,
    mk_(mk_EPId(-),id) -> let mk_(-,-,$st)=dfdsig(id) in
                            mk_(flowid,<READ>) in set  elems $st,
    mk_(nil, id)       -> let mk_($in,-,-) = dfdsig(id) in
                            flowid in set elems $in,
    mk_(id, nil) -> let mk_(-,out,-) = dfdsig(id) in
                      flowid in set elems out,
    mk_(fid,tid) -> let mk_(-,out,-) = dfdsig(fid),
                        mk_($in,-,-) = dfdsig(tid) in
                      (flowid in set  elems out) and 
                      (flowid in set  elems $in)
   end;
  
  IdsInSigsAvail : DSs * DFDTopo *  set of Signature -> bool 
  IdsInSigsAvail(dss,dfdtop,sigs) ==
  let fids=CollectExtDFs(dfdtop) in
  forall mk_($in,out,$st) in set sigs&
    elems $in subset  dom dfdtop and  
    elems out subset  dom dfdtop and  
    elems $st subset {mk_(id,m)|id in set dss union fids, 
                                m in set {<READ>,<READWRITE>}};
  
  LowerLevelUsed : DFDTopo * DFDMap -> bool 
  LowerLevelUsed(dfdtop,dfdmap) ==
  let ids =  dom dfdmap in
  forall mk_(fid,tid) in set  rng dfdtop &
   (is_DFDId(fid) => fid in set ids) and 
   (is_DFDId(tid) => tid in set ids);
  
  SigsForAllUsedIds : DFDId *  set of ([ProcId] * [ProcId]) * DFDMap *
                      DFDSig -> bool 
  SigsForAllUsedIds(id,top,dfdmap,dfdsig) ==
  (forall dfdid in set  dom dfdmap&
   let mk_(-,-,-,-,dfdsig')=dfdmap(dfdid) in
     dfdsig'(dfdid)=dfdsig(dfdid)) and
     let sigs= dom dfdsig in
       id in set sigs and -- dfds subset sigs and 
       forall mk_(fid,tid) in set top&
         ((is_MSId(fid) or is_DFDId(fid)) => (fid in set sigs)) and 
         ((is_MSId(tid) or is_DFDId(tid)) => (tid in set sigs));
  
  FlowConnectOK : ([ProcId] * [ProcId]) -> bool 
  FlowConnectOK(mk_(fid,tid)) ==
    ((is_EPId(fid) or fid=nil ) => (is_DFDId(tid) or is_MSId(tid))) and 
    ((is_EPId(tid) or tid=nil ) => (is_DFDId(fid) or is_MSId(fid)));
  
  NotRecursive :  set of ((DFDId|MSId) * (DFDId|MSId)) -> bool 
  NotRecursive(top) ==
    forall mk_(f,-) in set top&
     (f not in set TransClosure(f,top,{}));
  
  TransClosure : (DFDId|MSId) *  set of ((DFDId|MSId) * (DFDId|MSId)) *
                 set of (DFDId|MSId) ->  set of (DFDId|MSId)
  TransClosure(pid,top,$set) ==
   if exists mk_(fromid,toid) in set top&
      ((fromid=pid) or (fromid in set $set)) and (toid not in set $set)
   then let mk_(fromid,toid) in set top be st
           ((fromid=pid) or (fromid in set $set)) and 
           (toid not in set $set)
        in TransClosure(pid,top,$set union {toid})
   else $set

types
  Document =  set of Module;
  
  Module = ModuleId * Interface * Definitions;
  
  ModuleId = seq of char;
  
  Interface = Imports * Export;
  
  Imports =  set of Import;
  
  Import = ModuleId * ModuleSig;
  
  Export = ModuleSig;
  
  ModuleSig =  set of Sig;
  
  Sig = TypeSig|OpSig;
  
  TypeSig :: TypeId;
  
  TypeId = seq of char;
  
  OpSig :: id: Id 
           optype : OpType 
           stids : seq of Id;

  Definitions =  set of Definition;
  
  Definition = StateDef|OpDef --|... 
  ;
  
  StateDef :: id:Id
              fields: seq of Field;
  
  Field :: sel:[Id]
           type:Type;
  
  OpDef = ExplOp|ImplOp;
  
  ExplOp :: id:Id
            optype:OpType
            parms: seq of Pattern
            body:Stmt
            $pre:Expr;
  
  ImplOp :: id:Id
            partp: seq of ParType
            residtp:[IdType]
            $ext: seq of ExtVarInf
            body:ImplOpBody;
  
  ImplOpBody :: $pre:[Expr]
                $post:Expr;
  
  ParType :: pat:Pattern
             type:Type;
  
  IdType :: id:Id
            type:Type;
  
  ExtVarInf :: mode:ReadWriteMode
               id:Id
               type:Type;
  
  ReadWriteMode = <READ>|<READWRITE>;
  
  OpType :: dom':[Type]
            rng':[Type];
  
  Type = ProductType |MapType|SetType|SeqType | TypeId | BasicType |
         EnumType | OptionalType | UnionType --|... 
  ;
  
  ProductType :: product: seq1 of Type;
  
  MapType :: d: Type
             r: Type;
  
  SetType :: Type;
  
  SeqType :: Type;
  
  BasicType = <TOKEN> | <CHAR> | <BOOL>;
  
  EnumType :: seq of char;
  
  OptionalType :: Type;
  
  UnionType :: set of Type;
  Stmt = DclStmt|DefStmt|NonDetStmt|Call|Sequence|Return|<IDENT> -- |... 
  ;
  
  DclStmt :: dcls: set of AssDef
             body:Stmt;
  
  AssDef :: var:Id
            tp:Type;
  
  DefStmt :: lhs:Pattern
             rhs:Expr|Call
             $in:Stmt;
  
  NonDetStmt :: stmts: set of Stmt;
  
  Call :: oprt:Id
          args: seq of Expr;
  
  Sequence :: stmts: seq1 of Stmt;
  
  Return :: val:[Expr];
  
  Expr = LetExpr|IfExpr|QuantExpr|BinaryExpr|TupleConstructor|
         Apply|Id|BoolLit --| ... 
  ;
  
  LetExpr :: lhs:Pattern
             rhs:Expr
             $in:Expr;
  
  IfExpr :: test : Expr
            con  : Expr
            alt  : Expr;
  
  QuantExpr = ExistsExpr --| ... 
  ;
  
  ExistsExpr :: bind: MultTypeBind
  --ExistsExpr :: bind: seq1 of MultTypeBind
                pred:Expr;
  
  BinaryExpr :: left:Expr
                op:BinaryOp
                right:Expr;
  
  BinaryOp = <AND> | <OR> | <EQUAL> | <MEMB> --| ... 
  ;
  
  TupleConstructor :: fields: seq1 of Expr;
  
  Apply :: name:Expr
           arg: seq of Expr;
  
  BoolLit:: bool;
  
  MultTypeBind :: mtb: seq1 of TypeBind;
  
  TypeBind :: pats:seq of Pattern
              tp:Type;
  
  Pattern = PatternId|TuplePattern --| ... 
  ;
  
  PatternId :: name:[Id];
  
  TuplePattern :: fields: seq1 of Pattern;
  
  Id = seq of char


\newpage
\bibliographystyle{abbrv}
\bibliography{savdm} 

\addcontentsline{toc}{section}{Index}
\printindex

\end{document}
  
