\documentclass{article}
\usepackage{a4}
\usepackage{makeidx}
\usepackage{overture}

\newcommand{\StateDef}[1]{{\bf #1}}
\newcommand{\TypeDef}[1]{{\bf #1}}
\newcommand{\TypeOcc}[1]{{\it #1}}
\newcommand{\FuncDef}[1]{{\bf #1}}
\newcommand{\FuncOcc}[1]{#1}
\newcommand{\ModDef}[1]{{\tiny #1}}

\makeindex

\title{A crosswords assistant}
\author{Yves LEDRU\footnotemark}
\date{March 14, 1995}

\begin{document}
\maketitle
\footnotetext{Laboratoire de G\'enie Informatique - Institut IMAG (UJF - INPG-
CNRS), BP 
53x 38041 Grenoble
Cedex (FRANCE) - Tel + 33 76827214 - Fax + 33 76446675 - e-mail
Yves.Ledru@Imag.fr}


This tutorial example is taken out of a  VDM course given
to the students of the Dipl\^{o}me d'Etudes Sup\'erieures Sp\'ecialis\'ees en
G\'enie Informatique (5th year) at the Universit\'e Joseph Fourier.
This
example uses the implicit style of specification of VDM-SL and thus may not be
executed with the IFAD toolbox.


\section{Informal statement}

The crosswords assistant is a simple system which helps writing crosswords. 
Its user places words or black squares on a a crossword grid. 
The system  helps him keep a log of the words that appear on the grid.
These words  appear in a list of words (waiting list). 
The user will then informally check that these words effectively exist. Once a
word has been checked by the user, it will be validated and stored in a second
list. 

\subsection*{Example}

The user has placed sucessively the words {\tt word}, {\tt coal} and {\tt art}
on the grid. 

\vspace{0.5cm}

{\tt \begin{tabular}{r|c|c|c|c|c|c|c|c|}
 & 1 & 2 & 3 & 4 & 5 & 6 & 7 & 8\\
\hline
1 & & & & & & & & \\
\hline
2 & & & & & & & & \\
\hline
3 & & & &w& & & & \\
\hline
4 & & &c &o &a &l & & \\
\hline
5 & & &a &r&t & & & \\
\hline
6 & & & &d& & & & \\
\hline
7 & & & & & & & & \\
\hline
8 & & & & & & & & \\
\hline
\end{tabular}}

\vspace{0.5cm}

As a result, the words {\tt at} and {\tt ca} are also on the grid. The list of
words to validate is thus:

\begin{itemize}
\item[ ] words to validate : {\tt word, coal, art, at, ca}
\end{itemize}

The user then checks in his (paper) dictionary that all words but {\tt ca} are
english words. The two lists become:

\begin{itemize}
\item[ ] words to validate : {\tt ca}
\item[ ] valid words : {\tt word, coal, art, at}
\end{itemize}

In the sequel, the list of words to validate will also be referred to as the
``waiting list''.

\newpage
The user may now add the word {\tt cord} to the grid.

\vspace{0.5cm}

{\tt \begin{tabular}{r|c|c|c|c|c|c|c|c|}
 & 1 & 2 & 3 & 4 & 5 & 6 & 7 & 8\\
\hline
\multicolumn{9}{l}{\ldots}\\
\hline
3 & & & &w& & & & \\
\hline
4 & & &c &o &a &l & & \\
\hline
5 & & &a &r&t & & & \\
\hline
6 &c &o &r &d& & & & \\
\hline
\multicolumn{9}{l}{\ldots}\\
\end{tabular}}

\vspace{0.5cm}

{\tt ca} has now disappeared from the grid and the waiting list
is updated accordingly.

\begin{itemize}
\item[ ] words to validate : {\tt car, cord}
\item[ ] valid words : {\tt word, coal, art, at}
\end{itemize}

Other operations on the grid include:

\begin{itemize}
\item adding ``black'' squares
\item deleting some letters
\end{itemize}

The objective of the user is to fill in the whole grid with either black
squares or letters and to end up with an empty waiting list. 


values size : nat = 8;
       letters : set of char =
		{'a','b','c','d','e','f','g','h',
		'i','j','k','l','m','n','o','p',
		'q','r','s','t','u','v','w','x','y','z'};
	black : char = '*';
	white : char = '_'

types  word = seq of char 
	inv w == elems(w) subset letters 
		and len w >= 2  ;

        pos = nat1 
        inv pos_v == pos_v <= size;
	position :: h : pos
		    v : pos;

	grid = map position to char
	inv gr == rng gr subset (letters union {white, black})
		and dom gr = {mk_position(i,j) | i in set {1,...,size}, j in set {1,...,size}};

	HV = <H> | <V>

state  crosswords of
	 cwgrid : grid
	 valid_words : set of word
	 waiting_words : set of word
       inv mk_crosswords(gr,val,wait) == 
		CW_INVARIANT(gr,val,wait)
-- init mk_crosswords(gr,val,wait) ==
--        val = { } and wait = { } and 
--	forall i in set {1,...,size} &
--	forall j in set {1,...,size} &
--	gr(mk_position(i,j)) = white
end

functions

CW_INVARIANT: grid * set of word * set of word +> bool
CW_INVARIANT(gr,val,wait) ==
val inter wait = {} 
and WORDS(gr) subset (val union wait)
and wait subset WORDS(gr)
;

WORDS : grid +> set of word
WORDS(g) == HOR_WORDS(g) union VER_WORDS(g)
;

HOR_WORDS : grid +> set of word
HOR_WORDS(g) == dunion { WORDS_OF_SEQ(LINE(i,g)) | i in set {1,...,size}}
;
VER_WORDS : grid +> set of word
VER_WORDS(g) == dunion { WORDS_OF_SEQ(COL(i,g)) | i in set {1,...,size}}
;

LINE : pos *  grid +> seq of char
LINE(i,g) == [g(mk_position(i,c)) | c in set {1,...,size}]
;
COL : pos *  grid +> seq of char
COL(i,g) == [g(mk_position(l,i)) | l in set {1,...,size}]
;

WORDS_OF_SEQ : seq of char +> set of word
WORDS_OF_SEQ(s) == {w | w : word & 
		exists s1, s2 : seq of char &
			s = s1 ^ w ^ s2
		        and (s1 = [] or s1(len s1) = black or s1(len s1) = white)
			and (s2 = [] or s2(1) = black or s2(1) = white)}
;

COMPATIBLE : grid * word * position * HV +> bool
COMPATIBLE (g, w, p, d ) == 
	    (d = <H> => 
		(p.h + len w -1 <= size)
		and forall i in set inds w &
		    	g(mk_position(p.h + i -1, p.v)) = white
			or g(mk_position(p.h + i -1, p.v)) = w(i)
			)
	    and
	    (d = <V> => 
		(p.v + len w -1 <= size)
		and forall i in set inds w &
			g(mk_position(p.h, p.v + i -1)) = white
			or g(mk_position(p.h, p.v + i -1)) = w(i))
;

IS_LOCATED : grid * word * position * HV +> bool
IS_LOCATED (g, w, p, d ) == 
	    (d = <H> => 
		 forall i in set inds w &
		    g(mk_position(p.h + i -1, p.v)) = w(i))
	    and
	    (d = <V> => 
		 forall i in set inds w &
			g(mk_position(p.h, p.v + i -1)) = w(i))
;

IN_WORD: grid * position * HV +> bool
IN_WORD(g,p,d) ==
(d = <H> => 
	exists i,j : pos &
		i <= p.h and j >= p.h and i < j and
		forall k in set {i,..., j} &
			g(mk_position(k,p.v)) in set letters)
and
(d = <V> => 
	exists i,j : pos &
		i <= p.v and j >= p.v and i < j and
		forall k in set {i,..., j} &
			g(mk_position(p.h,k)) in set letters)

operations

VALIDATE_WORD (w : word)
ext wr valid_words : set of word
    wr waiting_words : set of word
   pre w in set waiting_words
  post valid_words = valid_words~ union {w}
       and waiting_words = waiting_words~ \ {w}
;

ADD_WORD (w : word, p : position, d : HV)
ext wr cwgrid : grid
    rd valid_words : set of word
    wr waiting_words : set of word
   pre COMPATIBLE(cwgrid, w, p, d)
  post (d = <H> =>
	cwgrid = cwgrid~ ++ {mk_position(p.h + i - 1, p.v) |-> w(i) | i in set inds w})
	and
 	(d = <V> =>
	cwgrid = cwgrid~ ++ {mk_position(p.h, p.v + i - 1) |-> w(i) | i in set inds w})
	and
	CW_INVARIANT(cwgrid, valid_words,waiting_words)
;

ADD_BLACK ( p : position)
ext wr cwgrid : grid
   pre cwgrid(p) = white
  post cwgrid = cwgrid~ ++ { p |-> black }
;

DELETE_BLACK ( p : position)
ext wr cwgrid : grid
   pre cwgrid(p) = black
  post cwgrid = cwgrid~ ++ { p |-> white }
;

STRONG_DELETE (w : word, p : position, d : HV)
ext wr cwgrid : grid
    rd valid_words : set of word
    wr waiting_words : set of word
   pre IS_LOCATED(cwgrid, w, p, d)
  post (d = <H> =>
	cwgrid = cwgrid~ ++ {mk_position(p.h + i - 1, p.v) |-> white | i in set inds w})
	and
 	(d = <V> =>
	cwgrid = cwgrid~ ++ {mk_position(p.h, p.v + i - 1) |-> white | i in set inds w})
	and
	CW_INVARIANT(cwgrid,valid_words, waiting_words)
;

SOFT_DELETE (w : word, p : position, d : HV)
ext wr cwgrid : grid
    rd valid_words : set of word
    wr waiting_words : set of word
   pre IS_LOCATED(cwgrid, w, p, d)
  post (d = <H> =>
	cwgrid = cwgrid~ ++ 
		{mk_position(p.h + i - 1, p.v) |-> white 
			| i in set inds w 
			& not IN_WORD(cwgrid~,mk_position(p.h + i - 1, p.v),<V>) })
	and
	(d = <V> =>
	cwgrid = cwgrid~ ++ 
		{mk_position(p.h, p.v + i - 1) |-> white 
			| i in set inds w 
			& not IN_WORD(cwgrid~,mk_position(p.h, p.v + i - 1),<H>) })
	and
	CW_INVARIANT(cwgrid,valid_words, waiting_words)


\end{document}
