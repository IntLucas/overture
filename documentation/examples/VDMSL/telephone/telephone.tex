\documentclass{article}
\usepackage{overture}%vdmsl-2e}
\usepackage{fullpage}

\title{A Telephone Exchange Specification in VDM-SL}
\author{Bernhard K. Aichernig}
\date{November 1998}

\begin{document}
\maketitle
The following example has been taken out of \cite{Abrial96} and has been 
translated from the B-notation into VDM-SL\cite{Fitzgerald&98, Jones90}. It
demonstrates how an event-based system may be modeled using the
specification language of the Vienna Development Method. In the following, 
operations specify the events which can be initiated either by  the system or 
by a subscriber (user). An implicit style using pre- and post-conditions has 
been chosen, in order to model the system's state transitions.

The model of the telephone
exchange is centred around a set of $subscribers$ who may be engaged in
telephone conversations through a network controlled by an exchange.
\section{Informal Specification}
As the communication between two subscribers is not installed immediately,
each subscriber navigates through a variety of statuses, which we study in
detail in what follows.
\begin{description}
\item[Free Subscribers] A subscriber might be free ($FR$),
meaning that he is not engaged in any telephone conversation or attempting to 
do so.
\item[Unavailable Subscribers] A subscriber who is temporarily unavailable
($UN$). Subscribers may enter this status as a result of a spontaneous 
decision on the part of the exchange. This happens when a subscriber has been
attempting to call another subscriber unsuccessfully for too long a period of
time. We may also enter this status at the end of some conversation between
two subscribers.
\item[Initiators or Recipients] Non-free and non-unavailable subscribers are
either initiators or recipients of telephone calls. An initiator is the only
one able to terminate a telephone conversation. By contrast, a recipient
hanging up just suspends the conversation, which may be resumed as soon as 
he again lifts his handset.
\item[Attempting, Waiting or Speaking Initiators]
An initiator may enter into various sub-states: he might be attempting ($AI$)
to call somebody (e.g. dialing), or waiting $WI$ for somebody to answer 
(connection established and the phone rings), or finally 
speaking $SI$ to somebody.
\item[Waiting or Speaking Recipients] A recipient might be speaking ($SR$) or
waiting ($WR$), because his own telephone is ringing, or because he has 
suspended an already engaged telephone conversation.   
\end{description}

\documentclass{article}
\usepackage{overture}%vdmsl-2e}
\usepackage{fullpage}

\title{A Telephone Exchange Specification in VDM-SL}
\author{Bernhard K. Aichernig}
\date{November 1998}

\begin{document}
\maketitle
The following example has been taken out of \cite{Abrial96} and has been 
translated from the B-notation into VDM-SL\cite{Fitzgerald&98, Jones90}. It
demonstrates how an event-based system may be modeled using the
specification language of the Vienna Development Method. In the following, 
operations specify the events which can be initiated either by  the system or 
by a subscriber (user). An implicit style using pre- and post-conditions has 
been chosen, in order to model the system's state transitions.

The model of the telephone
exchange is centred around a set of $subscribers$ who may be engaged in
telephone conversations through a network controlled by an exchange.
\section{Informal Specification}
As the communication between two subscribers is not installed immediately,
each subscriber navigates through a variety of statuses, which we study in
detail in what follows.
\begin{description}
\item[Free Subscribers] A subscriber might be free ($FR$),
meaning that he is not engaged in any telephone conversation or attempting to 
do so.
\item[Unavailable Subscribers] A subscriber who is temporarily unavailable
($UN$). Subscribers may enter this status as a result of a spontaneous 
decision on the part of the exchange. This happens when a subscriber has been
attempting to call another subscriber unsuccessfully for too long a period of
time. We may also enter this status at the end of some conversation between
two subscribers.
\item[Initiators or Recipients] Non-free and non-unavailable subscribers are
either initiators or recipients of telephone calls. An initiator is the only
one able to terminate a telephone conversation. By contrast, a recipient
hanging up just suspends the conversation, which may be resumed as soon as 
he again lifts his handset.
\item[Attempting, Waiting or Speaking Initiators]
An initiator may enter into various sub-states: he might be attempting ($AI$)
to call somebody (e.g. dialing), or waiting $WI$ for somebody to answer 
(connection established and the phone rings), or finally 
speaking $SI$ to somebody.
\item[Waiting or Speaking Recipients] A recipient might be speaking ($SR$) or
waiting ($WR$), because his own telephone is ringing, or because he has 
suspended an already engaged telephone conversation.   
\end{description}

\documentclass{article}
\usepackage{overture}%vdmsl-2e}
\usepackage{fullpage}

\title{A Telephone Exchange Specification in VDM-SL}
\author{Bernhard K. Aichernig}
\date{November 1998}

\begin{document}
\maketitle
The following example has been taken out of \cite{Abrial96} and has been 
translated from the B-notation into VDM-SL\cite{Fitzgerald&98, Jones90}. It
demonstrates how an event-based system may be modeled using the
specification language of the Vienna Development Method. In the following, 
operations specify the events which can be initiated either by  the system or 
by a subscriber (user). An implicit style using pre- and post-conditions has 
been chosen, in order to model the system's state transitions.

The model of the telephone
exchange is centred around a set of $subscribers$ who may be engaged in
telephone conversations through a network controlled by an exchange.
\section{Informal Specification}
As the communication between two subscribers is not installed immediately,
each subscriber navigates through a variety of statuses, which we study in
detail in what follows.
\begin{description}
\item[Free Subscribers] A subscriber might be free ($FR$),
meaning that he is not engaged in any telephone conversation or attempting to 
do so.
\item[Unavailable Subscribers] A subscriber who is temporarily unavailable
($UN$). Subscribers may enter this status as a result of a spontaneous 
decision on the part of the exchange. This happens when a subscriber has been
attempting to call another subscriber unsuccessfully for too long a period of
time. We may also enter this status at the end of some conversation between
two subscribers.
\item[Initiators or Recipients] Non-free and non-unavailable subscribers are
either initiators or recipients of telephone calls. An initiator is the only
one able to terminate a telephone conversation. By contrast, a recipient
hanging up just suspends the conversation, which may be resumed as soon as 
he again lifts his handset.
\item[Attempting, Waiting or Speaking Initiators]
An initiator may enter into various sub-states: he might be attempting ($AI$)
to call somebody (e.g. dialing), or waiting $WI$ for somebody to answer 
(connection established and the phone rings), or finally 
speaking $SI$ to somebody.
\item[Waiting or Speaking Recipients] A recipient might be speaking ($SR$) or
waiting ($WR$), because his own telephone is ringing, or because he has 
suspended an already engaged telephone conversation.   
\end{description}

\documentclass{article}
\usepackage{overture}%vdmsl-2e}
\usepackage{fullpage}

\title{A Telephone Exchange Specification in VDM-SL}
\author{Bernhard K. Aichernig}
\date{November 1998}

\begin{document}
\maketitle
The following example has been taken out of \cite{Abrial96} and has been 
translated from the B-notation into VDM-SL\cite{Fitzgerald&98, Jones90}. It
demonstrates how an event-based system may be modeled using the
specification language of the Vienna Development Method. In the following, 
operations specify the events which can be initiated either by  the system or 
by a subscriber (user). An implicit style using pre- and post-conditions has 
been chosen, in order to model the system's state transitions.

The model of the telephone
exchange is centred around a set of $subscribers$ who may be engaged in
telephone conversations through a network controlled by an exchange.
\section{Informal Specification}
As the communication between two subscribers is not installed immediately,
each subscriber navigates through a variety of statuses, which we study in
detail in what follows.
\begin{description}
\item[Free Subscribers] A subscriber might be free ($FR$),
meaning that he is not engaged in any telephone conversation or attempting to 
do so.
\item[Unavailable Subscribers] A subscriber who is temporarily unavailable
($UN$). Subscribers may enter this status as a result of a spontaneous 
decision on the part of the exchange. This happens when a subscriber has been
attempting to call another subscriber unsuccessfully for too long a period of
time. We may also enter this status at the end of some conversation between
two subscribers.
\item[Initiators or Recipients] Non-free and non-unavailable subscribers are
either initiators or recipients of telephone calls. An initiator is the only
one able to terminate a telephone conversation. By contrast, a recipient
hanging up just suspends the conversation, which may be resumed as soon as 
he again lifts his handset.
\item[Attempting, Waiting or Speaking Initiators]
An initiator may enter into various sub-states: he might be attempting ($AI$)
to call somebody (e.g. dialing), or waiting $WI$ for somebody to answer 
(connection established and the phone rings), or finally 
speaking $SI$ to somebody.
\item[Waiting or Speaking Recipients] A recipient might be speaking ($SR$) or
waiting ($WR$), because his own telephone is ringing, or because he has 
suspended an already engaged telephone conversation.   
\end{description}

\include{generated/latex/specification/telephone.vdmsl}

\section{Typical Scenarios}
Finally, some typical sequences of valid events are listed:\\ \\
Lift $\longrightarrow$ ClearAttempt \\
\\
Lift $\longrightarrow$ MakeUn $\longrightarrow$ ClearUn \\
\\
Lift $\longrightarrow$ Connect $\longrightarrow$ ClearWait\\
\\
Lift $\longrightarrow$ Connect $\longrightarrow$ Answer $\longrightarrow$ ClearSpeak $\longrightarrow$ ClearUn\\
\\
Lift $\longrightarrow$ Connect $\longrightarrow$ Answer $\longrightarrow$ Suspend $\longrightarrow$ ClearWait
\bibliographystyle{plain}
\bibliography{telephone}
\end{document}


\section{Typical Scenarios}
Finally, some typical sequences of valid events are listed:\\ \\
Lift $\longrightarrow$ ClearAttempt \\
\\
Lift $\longrightarrow$ MakeUn $\longrightarrow$ ClearUn \\
\\
Lift $\longrightarrow$ Connect $\longrightarrow$ ClearWait\\
\\
Lift $\longrightarrow$ Connect $\longrightarrow$ Answer $\longrightarrow$ ClearSpeak $\longrightarrow$ ClearUn\\
\\
Lift $\longrightarrow$ Connect $\longrightarrow$ Answer $\longrightarrow$ Suspend $\longrightarrow$ ClearWait
\bibliographystyle{plain}
\bibliography{telephone}
\end{document}


\section{Typical Scenarios}
Finally, some typical sequences of valid events are listed:\\ \\
Lift $\longrightarrow$ ClearAttempt \\
\\
Lift $\longrightarrow$ MakeUn $\longrightarrow$ ClearUn \\
\\
Lift $\longrightarrow$ Connect $\longrightarrow$ ClearWait\\
\\
Lift $\longrightarrow$ Connect $\longrightarrow$ Answer $\longrightarrow$ ClearSpeak $\longrightarrow$ ClearUn\\
\\
Lift $\longrightarrow$ Connect $\longrightarrow$ Answer $\longrightarrow$ Suspend $\longrightarrow$ ClearWait
\bibliographystyle{plain}
\bibliography{telephone}
\end{document}


\section{Typical Scenarios}
Finally, some typical sequences of valid events are listed:\\ \\
Lift $\longrightarrow$ ClearAttempt \\
\\
Lift $\longrightarrow$ MakeUn $\longrightarrow$ ClearUn \\
\\
Lift $\longrightarrow$ Connect $\longrightarrow$ ClearWait\\
\\
Lift $\longrightarrow$ Connect $\longrightarrow$ Answer $\longrightarrow$ ClearSpeak $\longrightarrow$ ClearUn\\
\\
Lift $\longrightarrow$ Connect $\longrightarrow$ Answer $\longrightarrow$ Suspend $\longrightarrow$ ClearWait
\bibliographystyle{plain}
\bibliography{telephone}
\end{document}
