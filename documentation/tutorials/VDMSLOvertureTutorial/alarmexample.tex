\appendix

\chapter{A Chemical Plant Example}\label{app:alarm}

This appendix presents the requirements for a simple alarm system for a
chemical plant. It forms a running example that serves to illustrate
the process described earlier and to introduce elements of the VDM-SL
modelling language. Although the modelling process is described here
as though it were a single-pass activity, a real development would
usually be iterative. 


\section{An informal description}



The example is inspired by a subcomponent of a large
alarm system developed by IFAD A/S and introduced in 
\cite{Fitzgerald&98b}. 
Chapter~\ref{cha:toolbox} provides an interactive and hands-on tour of
the tools available for supporting the development of the model.

Imagine that you are developing a system that manages the calling out
of experts to deal with operational faults discovered in a chemical
plant.  The plant is equipped with sensors that are able to raise
alarms in response to conditions in the plant.  When an alarm is
raised, an expert must be called to the scene.  Experts have different
qualifications for coping with different kinds of alarms. It has been
decided to produce a model to ensure that the rules
concerning the duty schedule and the calling out of experts are
correctly understood and implemented. The individual requirements are
labelled R1, R8 for further reference:

\begin{reqs}
\item A computer-based system is to be developed to manage the alarms 
of this plant.
\item Four kinds of qualifications are needed to cope with the alarms: 
 electrical, mechanical, biological, and chemical.
\item There must be experts on duty during all periods allocated in 
the system.
\item Each expert can have a list of qualifications.
\item Each alarm reported to the system has a qualification associated 
with it along with a description of the alarm that can be understood 
by the expert.
\item Whenever an alarm is received by the system an expert with the 
right qualification should be found so that he or she can be 
paged.
\item The experts should be able to use the system database to check 
when they will be on duty.
\item It must be possible to assess the number of experts on duty.
\end{reqs}

In the next section the development of a model of an alarm
system to meet these requirements is initiated. The purpose of the model is to
clarify the rules governing the duty roster and calling out of experts
to deal with alarms.

\section{A VDM-SL model of the Alarm example}\label{sec:VDMModel}

This section presents the full VDM-SL model
of the alarm example. However, it does so without any explanatory
text. That is placed in the VDM-SL book so if you are a newcommer to
VDM-SL please read that there.

\begin{vdmsl}
types

  Plant :: schedule : Schedule
           alarms   : set of Alarm
  inv mk_Plant(schedule,alarms) ==
        forall a in set alarms &
	   forall peri in set dom schedule &
	     QualificationOK(schedule(peri),a.quali);
	     
  Schedule = map Period to set of Expert
inv sch ==
   forall exs in set rng sch &
          exs <> {} and
          forall ex1, ex2 in set exs &
                 ex1 <> ex2 => ex1.expertid <> ex2.expertid;

  Period = token;

  Expert :: expertid : ExpertId
            quali    : set of Qualification
  inv ex == ex.quali <> {};

  ExpertId = token;

  Qualification = <Elec> | <Mech> | <Bio> | <Chem>;
	   
  Alarm :: alarmtext : seq of char
           quali     : Qualification
\end{vdmsl}

The functionality from the requirements presented above can be defined
in a number of functions as follows.

\begin{vdmsl}
functions

  NumberOfExperts: Period * Plant -> nat
  NumberOfExperts(peri,plant) ==
    card plant.schedule(peri)
  pre peri in set dom plant.schedule;

  ExpertIsOnDuty: Expert * Plant -> set of Period
  ExpertIsOnDuty(ex,mk_Plant(sch,-)) ==
    {peri| peri in set dom sch & ex in set sch(peri)};

  ExpertToPage(a:Alarm,peri:Period,plant:Plant) r: Expert
  pre peri in set dom plant.schedule and
      a in set plant.alarms
  post r in set plant.schedule(peri) and
       a.quali in set r.quali;

  QualificationOK: set of Expert * Qualification -> bool
  QualificationOK(exs,reqquali) ==
    exists ex in set exs & reqquali in set ex.quali
\end{vdmsl}

The \texttt{ChangeExpert} function below is not correct but it is used
for exercise/test purposes in this tutorial.

\begin{vdmsl}
functions

ChangeExpert: Plant * Expert * Expert * Period -> Plant
ChangeExpert(mk_Plant(plan,alarms),ex1,ex2,peri) ==
  mk_Plant(plan ++ {peri |-> plan(peri)\{ex1} union {ex2}},
           alarms)
\end{vdmsl}

In order to test the model different values can be defined. Such value
definitions make use of the types defined in the VDM-SL model.


\begin{vdmsl}
values
 
  p1:Period = mk_token("Monday day");
  p2:Period = mk_token("Monday night");
  p3:Period = mk_token("Tuesday day");
  p4:Period = mk_token("Tuesday night");
  p5:Period = mk_token("Wednesday day");
  ps : set of Period = {p1,p2,p3,p4,p5};

  eid1:ExpertId = mk_token(134);
  eid2:ExpertId = mk_token(145);
  eid3:ExpertId = mk_token(154);
  eid4:ExpertId = mk_token(165);
  eid5:ExpertId = mk_token(169);
  eid6:ExpertId = mk_token(174);
  eid7:ExpertId = mk_token(181);
  eid8:ExpertId = mk_token(190);
  
  e1:Expert = mk_Expert(eid1,{<Elec>});
  e2:Expert = mk_Expert(eid2,{<Mech>,<Chem>});
  e3:Expert = mk_Expert(eid3,{<Bio>,<Chem>,<Elec>});
  e4:Expert = mk_Expert(eid4,{<Bio>});
  e5:Expert = mk_Expert(eid5,{<Chem>,<Bio>});
  e6:Expert = mk_Expert(eid6,{<Elec>,<Chem>,<Bio>,<Mech>});
  e7:Expert = mk_Expert(eid7,{<Elec>,<Mech>});
  e8:Expert = mk_Expert(eid8,{<Mech>,<Bio>});
  exs : set of Expert = {e1,e2,e3,e4,e5,e6,e7,e8};

  s: map Period to set of Expert
     = {p1 |-> {e7,e5,e1},
        p2 |-> {e6},
        p3 |-> {e1,e3,e8},
        p4 |-> {e6}};

  a1:Alarm = mk_Alarm("Power supply missing",<Elec>);
  a2:Alarm = mk_Alarm("Tank overflow",<Mech>);
  a3:Alarm = mk_Alarm("CO2 detected",<Chem>);
  a4:Alarm = mk_Alarm("Biological attack",<Bio>);
  alarms: set of Alarm = {a1,a2,a3,a4};

  plant1: Plant = mk_Plant(s,{a1,a2,a3,a4})
\end{vdmsl}  

A basic explicit operation for test purposes can be defined as below.

\begin{vdmsl}  
operations

Run: Expert ==> set of Period
Run(e) == return ExpertIsOnDuty(e, plant1);
\end{vdmsl}

In the new VDM-10 variant of VDM-SL {\bf\ttfamily traces} have been
incorporated since they can be used with tool support for
combinatorial testing purposes.

%\begin{minipage}{\linewidth}
\begin{vdmsl}
traces

  Test1: let a in set alarms
         in
           let p in set ps 
           in
             (NumberOfExperts(p,plant1);
              pre_ExpertToPage(a,p,plant1);
              let ex in set exs
              in
                post_ExpertToPage(a,p,plant1,ex))
               
  Test2: let ex in set exs
         in
           ExpertIsOnDuty(ex,plant1)
\end{vdmsl}  
%\end{minipage}
