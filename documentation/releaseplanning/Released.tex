
\chapter{Released}


%
% Release 0.2.0
%
\begin{enumerate}

\item[Release 0.2.0 (March 2010):] The aim for this release is to
  include the following features:  
\begin{enumerate}
\item \markDone{Analyse whether we can go for a debugger without
  DLTK. \developer{AugustoRibeiro},\textbf{Dependencies: None}\label{analyseDLTKDebug}}
 
\item \markDone{ The combinatorial testing feature will be updated in the IDE
  with the features for reducing the size of the test suites in
  intelligent ways made available in VDMJ for all VDM
  dialects. \developer{KennethLausdahl}, \textit{Done(Prototype):
    2. feb 2010 - Kenneth}}

\item \markDone{In the edit perspective automatic type checking will only be
  performed in case the syntax is correct for all definitions such
  that type errors does not get mixed with syntax
  errors. \developer{KennethLausdahl}, \textit{Done: 1. feb 2010 - Kenneth} } 

\item \markDone{ When the Overture tool is first started up and the user kills
  the welcome window the VDM++ edit perspective will be started
  automatically. \developer{AugustoRibeiro}, \textit{Done: 1. feb 2010 - Kenneth}} 

\item \markDone{Syntax and type check the arguments used in the debug
  configuration. \developer{KennethLausdahl}}, \textbf{Dependencies: None} \textbf{Remarks:} missing doc

\item \markDone{ Remove all communication exceptions from VDMJ which causes the
  console to change (e.g.\ the existing problem with the debugger in
  the IDE whenever an invariant for instance invariants are
  broken). \developer{NickBattle, KennethLausdahl}, \textbf{Dependencies: None}}

\end{enumerate}

%\item Specify the top level scheduling strategy for the interpreter
%  dealing with multiple threads in VDM++ and VDM-RT (to subsequently
%  be implemented in the interpreter).\\
%  \hfill \developer{KennethLausdahl,++} \label{toplevelscheduling}

\item \markDone{Make the pretty printer with test coverage usable for all VDM
  dialects. \developer{KennethLausdahl, NickBattle}} 



\end{enumerate}




%
% Release 0.3.0
%
\begin{enumerate}
\item[Release 0.3.0 (June 2010):] The aim for this release is to
  include the following features:  
\begin{enumerate}


\item  \markDone{Decide if we want to include the Update Manager or embed a number of videly used features like Sub Version control.: We will include the update manager and self update for RCP Update to reduce the effort needed to update the tool}

\item  \markDone{Replace DLTK. This includes a complete recoding of all existing Editors, Wizards, Outlines etc including a build from scratch Debug interface inspired from DLTK but only using the DBGProtocol as DLTK do.\footnote{The decision for replacing DLTK has been made based on the investigation of Eclipse where it have been found more feasible to use Eclipse directly rather than using DLTK since a lot of the existing code do work rounds to get back to eclipse. The benefit of this change in the long run is seen larger than the work to recreate some of the DLTK features.} (For lower level description of the goals please refer to appendix \ref{app:lowlevelgoals}). \developer{AugustoRibeiro,KennethLausdahl}, \textbf{Dependencies: \ref{picturesEclipse}} }

\item  \markDone{Launch configuration setup: 
	\begin{enumerate}
		\item When creating a new debug configuration it shall be named after
  the name of the project (and not just ``New configuration'').
		\item The configuration should be linked to the project so it is available from \textit{Debug As}
		\item General plugin.xml problem.
	\end{enumerate} \developer{AugustoRibeiro,KennethLausdahl}, \textbf{Dependencies: R020:\ref{analyseDLTKDebug}}\label{launchconfig}}

\item \markDone{Making sure that the run and debug buttons in the IDE work
  appropriately and does not crash whenever the user tries to start
  them. Ideally these should simply change to the debug perspective if
  no debug configuration is present and be ready to type an expression
  into a quick console where simple expressions can be typed and
  executed (and up and down in the list of commands given is
  supported). \developer{KennethLausdahl}, \textbf{Dependencies: \ref{launchconfig}} }



\item  \markDone{Stop all threads when a breakpoint or a run-time error has
  occured in one of the threads. \developer{NickBattle,KEL, ARI}, \textbf{Dependencies: R020:\ref{toplevelscheduling}} }

\item  \markDone{In the debug perspective it shall be possible to look into the
  contents of objects in the variables view. \developer{KEL}, \textbf{Dependencies: R020:\ref{analyseDLTKDebug},\ref{completeAST}} }



\item \markDone{ When creating new projects (by wizard) in the IDE include the possibility
  for including standard libraries (i.e.\ (VDM headers) IO and Math) in the
  project. These are then to be included in a special ``lib''
  directory. \developer{??}, \textbf{Dependencies: None}} 

\item  \markDone{ Make sure that proof obligations for standard libraries are
  ignorred. \developer{NickBattle}, \textbf{Dependencies: None}  }

\item \markDone{Make it possible for users to provide static implementations to ``is not yet specified'' functions and
  operations by adding a Jar file to a lib folder in the project. \developer{KennethLaudahl,NickBattle}\label{staticIsNotYetSpecified} \textit{Prototype done}}

\item \markDone{ Make it possible for users to add a java
  implementation of a VDM class by adding a jar to the lib folder. The
  implementation should be done so one instance in java corresponds to
  one instance in VDM. No
  static. \developer{KennethLausdahl,NickBattle},
  \textbf{Dependencies: \ref{staticIsNotYetSpecified}} }

\item \markDone{ VDM Tools
	\begin{enumerate}
		\item  Create a VDM Tools project file from a Overture project and launch the VDM Tools GUI.\developer{KEL}
	\end{enumerate}}

\item  \markDone{Coverage Editor\developer{KEL}}

\item  \markDone{DLTK replacement

	\begin{enumerate}
		\item Recreate Debugger. New debug protocol separation from VDMJC. Launch, Debug runner, 
		\item Recreate the parser
		\item Recreate the builder
		\item Recreate the Editor
		\item Recreate the outline
		\item Recreate the package explorer
	\end{enumerate}}

\end{enumerate}
\end{enumerate}
