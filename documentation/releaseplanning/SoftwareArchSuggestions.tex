\documentclass{overturerep}
\usepackage{url}
\usepackage{graphics}
\usepackage{times}
\usepackage{listings}
%\usepackage{color}
\usepackage[usenames,dvipsnames]{color}
\usepackage{graphicx}
\usepackage{latexsym}
\usepackage{longtable} % ,multirow}
\usepackage{amsmath}
\usepackage{amsthm}
\usepackage{hyperref}
\newtheorem{sug}[subsection]{Suggestion}

\title{Suggestions for Software Architecture in the Overture tool suite}
\author{Rasmus Lauritsen \and Luis D. M. D. Couto}
\reportno{SG-04-2013\footnote{SG is for suggestion}}
\begin{document}

\maketitle

\begin{abstract}
  The success of Overture has provided the community with a dependable
  tool. To bring state of the art formal methods to this community
  Overture becomes a platform for a variety of formal methods tools.
  Therefore, its software architecture needs constantly to support
  extensibility and maintainability.
\end{abstract}

\chapter{Introduction}

\section{Code Style}
A strong asset with maintainable code is normalisation such that
developers are familiar with the code-style being used. The first
suggestion is thus to adopt a code-style for Overture that at least
makes it follow the lines stipulated in
\url{http://www.oracle.com/technetwork/java/javase/documentation/codeconvtoc-136057.html}.

\begin{sug}
  Adopt a code-style for Overture, using those adopted by Google is an
  option.
  \url{http://code.google.com/p/java-coding-standards/wiki/CodeOrganisation}
\end{sug}

It is important that the code-style is advocated by all community
members and developers. It should not only be document but also be
explicitly visible in the code.

\begin{sug}
  Advocate code-style and make it visible in the source. Consider
  periodic reviews of core components.
\end{sug}

A central part of good code-style is documentation. All core components
in Overture should be well documented in the source-code such that a
coherent understanding of the code is obtainable from the source files
alone.

\begin{sug}
Thorough documentation of the code in the source files should be available.
\end{sug}

However, knowing how classes-work by them self is typically not enough
to understand a software project as complex as Overture.  Thus a
description of the intended interplay between classes and which tasks
they perform is needed. A well scoped package typically has a public
factory class or similar point of access.

\begin{sug}
  Describe the intended interplay between \emph{public}
  classes/interfaces and intended use of a coherent package in its
  public entry point classes.
\end{sug}

\section{Packages}
The package structure for Overture is used for internal modularisation
which is conflicting with the scoping mechanisms in Java. Packages
were never intended to be used that way. The proper way to have
internal modularisation is through source folders.

\begin{sug}
  Reduce Overture to a few packages with high cohesion. The
  modularisation reflected in existing packages is good but should be
  realised through source folders.
\end{sug}

With a few well defined packages it becomes feasible to use
package-scoping to create a narrow well scoped interface exposed by
the Overture code base.

\begin{sug}
  Reduce scoping on classes to package scope where possible. 
\end{sug}

An additional benefit with this is that the Eclipse OSGi framework
will have small and well defined manifest files; only a few packages
needs to be listed in these. New developments would seldom introduce
new packages and thus this also result in infrequent manifest file
updates. In turn this might reduce the number of mysterious OSGi
errors following such updates.

\begin{sug}
  Include small examples of component usage for each package.
\end{sug}

Showing a small example of how the public classes and interfaces from
a package is intended to be used speeds up future development
significantly. The task of modifying an example to do something is
typically much easier and gives more confidence in the code than
taking a bunch of public classes and putting them together only from
documentation. At least the gives the confidence to a new developer
that the original ideas are maintained.

\end{document}