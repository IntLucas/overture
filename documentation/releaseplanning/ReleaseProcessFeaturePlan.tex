% -*- mode: latex; mode: flyspell; coding: utf-8 -*-
\documentclass{overturerep}

\usepackage{hyperref}
\usepackage{times}
\usepackage[scaled]{helvet}
\usepackage[hmargin=24mm,vmargin=36mm]{geometry}


\def\today{\number\day\space\ifcase\month\or January\or February\or March\or April\or May\or June\or July\or August\or September\or October\or November\or December\fi\space\number\year} % DD Month YYYY

\newenvironment{denseitemize}
  {\begin{itemize}\setlength{\itemsep}{0pt}\setlength{\parskip}{0pt}\setlength{\parsep}{0pt}}
  {\end{itemize}}
\newenvironment{denseenumerate}
  {\begin{enumerate}\setlength{\itemsep}{0pt}\setlength{\parskip}{0pt}\setlength{\parsep}{0pt}}
  {\end{enumerate}}

\newcommand{\developer}[1]{{\scriptsize \fbox{#1}}}


\setlength{\parindent}{0pt}
\setlength{\parskip}{\medskipamount}

\title{Overture Release Process \& Feature List}
\author{Joey W. Coleman}
%\reportno{TR-2011-??}

\hypersetup{linkcolor=blue,filecolor=blue,urlcolor=blue,colorlinks=true} 

\begin{document}
\maketitle
\tableofcontents

%%%%%%%%%%%%%%%%%%%%%%%%%%%%%%%%%%%%%%%%%%%%%%%%%%%%%%%%%%%%%%%%
%% Release Process
%%%%%%%%%%%%%%%%%%%%%%%%%%%%%%%%%%%%%%%%%%%%%%%%%%%%%%%%%%%%%%%%


\chapter{Release Process}

  This section is an attempt to explain a process and rationale for
  the release process of the Overture tool.  Much of the information
  in this document should be obvious, however, stating it explicitly
  is useful to reduce ambiguity in discussions.

\section{Version Numbering}

  Version numbers are of the generally form $x.y.z$, e.g. 1.1.2, and
  in the run-up to a non-trivial release, may include a release
  candidate suffix, e.g. 1.2.0-RC1.  The meaning of each position is
  as follows:

\begin{description}
\item[Major, x.\_.\_] Indicates major changes to the tool, including
  replacement of a foundational subsystem (such as the AST), or the
  addition of a major new feature set (such as the possible future
  merge of the COMPASS tool).

  This sort of change requires as much testing as we can manage and
  must use release candidates prior to release.

\item[Minor, \_.x.\_] Indicates either the addition of significant
  (but not foundational) new features or replacement of a minor
  subsystem.  Adding a feature such as code completion is an example.

  This feature requires thorough testing and should use release
  candidates.

\item[Increment, \_.\_.x] Indicates an apparently regression-free bugfix
  release.  The only changes relative to the prior version are bug
  fixes and very minor feature changes.

  In order to allow these release to happen quickly, this sort of
  release should not use release candidates.  However, it must pass
  the unit test suite and it must have had the manual GUI testing
  procedure checked.

\item[Release Candidates, x.y.\_-RCn] Indicates a version that, if no
  problems are found, could become the version indicated (1.2.0, in
  this case).

  Any release candidate should pass the unit tests and manual GUI
  testing procedure, but this is not guaranteed.  It must pass all
  tests before becoming a full release.
\end{description}


\section{Development with Git}
  One of the primary features of Git is its ability to cheaply create,
  maintain, and re-integrate many branches of development within a
  repository.  The actual semantic work of merging is no cheaper than
  other version control systems --nor is it likely that anything can
  make it cheaper-- but the cost of tracking files and parents is
  minimized through technical support.

  So, on that basis, every develop should commit changes to their own
  personal branch on sourceforge, and contact the Release Manager when
  the feature(s) they are working on are ready to be merged into the
  main `master' branch.  Around that point they should first pull from
  the master branch into their own branch and perform the integration
  before their changes are merged back into the master branch.


\section{Testing}

  Testing, at present, is comprised of a few procedures:

\begin{denseenumerate}
\item an incomplete set of unit tests; and,
\item an informal procedure for exercising the tool's features using the GUI; and,
\item distributing the tool to many users to see if it works for them.
\end{denseenumerate}

  Clearly, we need to work on this.

\section{Bug Tracking \& Feature Requests}

  For Overture in general we use SourceForge to track both, and for
  DESTECS we use ChessForge.  COMPASS will likely use SourceForge as
  well.

  I am strongly considering using a Trac instance to track bugs,
  features, and releases, if only to have some automatic dependency
  tracking.

\section{Documentation}

  Much more ought to be said about this.

  Every release should be accompanied by release notes, the contents
  of which are yet to be fully determined.

  We need a complete build guide that lists the development tools
  used, their role in the process (and any deviations from
  ``standard'' usage by us), and how to invoke the process in various
  contexts (release building, local building, etc).  This is in
  preparation, and will end up on the wiki (at least).

  This document itself is intended to be a living reference, as part
  of this it will be updated regularly.  Furthermore, I will be
  investigating tools like Trac to produce a task list that records
  dependencies, release versions, and so on.



%%%%%%%%%%%%%%%%%%%%%%%%%%%%%%%%%%%%%%%%%%%%%%%%%%%%%%%%%%%%%%%%
%% Feature Plan
%%%%%%%%%%%%%%%%%%%%%%%%%%%%%%%%%%%%%%%%%%%%%%%%%%%%%%%%%%%%%%%%

\chapter{Feature Plan}

  This section presents the list of planned features and an
  approximate timeline of when they are expected to be in an official
  release of the Overture/DESTECS/COMPASS Tool.  The intention is to
  cover the development of all three tools, and the feature planning
  for Overture does, occasionally, rely on the context of DESTECS and
  (eventually) COMPASS development.

  The feature plan is fairly sparse, but represents the things that
  are firmly expected.

  It's worth noting that I hope to release minor versions of the
  various tools regularly, so those are not explicitly recorded unless
  it's relevant.

\section{Actions Timeline}
\label{sec:actions}

\paragraph{February 2012}
\begin{denseitemize}
\item Overture 1.2.1 release, with minor bugfixes \developer{ARI, KEL}
\item DESTECS 1.3.0 release, based on Overture 1.2.1 \developer{ARI, KEL}
\end{denseitemize}

\paragraph{March 2012} 
\begin{denseitemize}
\item DESTECS 1.3.1 release, based on Overture 1.2.1 \developer{ARI, KEL}
\item Overture v2 release, based on New AST \developer{KEL, JWC}
\end{denseitemize}

\paragraph{April 2012}
\begin{denseitemize}
\item DESTECS 1.3.2 release, based on Overture 1.2.1 \developer{ARI, KEL}
\end{denseitemize}

\paragraph{May 2012} 
\begin{denseitemize}
\item DESTECS 1.3.3 release, based on Overture 1.2.1 \developer{ARI}
\item DESTECS development freeze
\end{denseitemize}


\section{Language Board ``execution'' RMs}

  These features and changes come from the Overture Language Board and
  represent things that have change and are now a part of the VDM
  language family standards.

  Language board Requests for Modification will be added to this list
  after they reach ``execution'' phase.  As and when people pick up
  these tasks, they will be added to the action timeline in
  Section~\ref{sec:actions}; they will be removed when they have been
  completed.

\begin{tabular}{clcc}
SF.net ID & Title & Who & When \\
\hline
\href{https://sourceforge.net/tracker/?func=detail&aid=3011828&group_id=141350&atid=1127184}{3011828}
	& Include the non-deterministic statement inside traces
	& Unknown & Unknown 
\\
\href{https://sourceforge.net/tracker/?func=detail&aid=3220182&group_id=141350&atid=1127184}{3220182}
	& Expressions in periodic thread definitions
	& Nick B & in Git
\\
\href{https://sourceforge.net/tracker/?func=detail&aid=3220223&group_id=141350&atid=1127184}{3220223}
	& Values in duration / cycles statements	
	& Nick B & in Git
\\
\href{https://sourceforge.net/tracker/?func=detail&aid=3220437&group_id=141350&atid=1127184}{3220437}
	& Extend duration and cycles	
	& Unknown & Unknown
\\
\href{https://sourceforge.net/tracker/?func=detail&aid=3438625&group_id=141350&atid=1127184}{3438625}
	& Append Map Pattern
	& Unknown & Unknown
\end{tabular}



\section{Other Features}
This section pulls in the old release planning document and adds some
that don't yet have specific work plans.  Some may no longer be
relevant.

\begin{denseenumerate}
\item Complete the new ASTGen utility and add all sources for it to
  the tools utility in the Overture SF SVN. \developer{NickBattle} \textbf{Dependencies: None} \label{completeAST}
  
\item Include VDMDoc (inspired by JavaDoc) for all VDM dialects. / Allowing multi line comments in VDM to be represented in the AST \developer{??}

\item Make VDM ast pretty printer for math syntax
   
\item Documentation of the Overture IDE in order to enable more
  developers to make updates for the IDE 
  development to follow similar principles. This shall document the
  implementation of the editor, the parser, the builder, the outline
  and the AST access. \textbf{Status:} Ongoing \developer{AugustoRibeiro}, \textbf{Dependencies: \ref{completeAST}}
    \begin{denseenumerate}
    \item Parser 
    \item Builder; How to insure TC is completed
    \item How to add debug with a new interpreter and corresponding launch configuration.
    \end{denseenumerate}

\item Provide help (F1) inside Eclipse

\item Provide goto definition and completion; \developer{AugustoRibeiro}

\item Change to the AST generated from the new version of ASTGen all
  over the Overture sources (i.e.\ this will affect the development of
  UML and POtrans components). \developer{NickBattle,
    KennethLausdahl}, \textbf{Dependencies: \ref{completeAST}}

\item Quick fix incorporated in the editor view of the IDE for all VDM
  dialects. \developer{??}

\item Goto definition in the edit view of the IDE across all files in
  a project for all VDM dialects. \developer{??}

\item First version of refactoring support for all VDM
  dialects. \developer{??}
 
\item Stable version of the UML mapper enabling proper round-trip
  engineering between VDM and UML class diagrams and enable generation
  of traces from a subset of UML sequence
  diagrams. \developer{KennethLausdahl,PeterGormLarsen}

\item First release of code generator from VDM to a programming
  language. \developer{MarcelVerhoef}

\item Increased \LaTeX\ support in the editor such that parts outside
  the \texttt{vdm\_al} environments are highlighed as TeXClipse would
  do it. \developer{??}
  \item Make Java code generator

\item Make C code generator

\item Make C++ code generator

\item Make VDM Doc parser

\item Make VDM source format
  
\item Tests of the plug-ins

\item Make a VDM model of Multi-core CPU distribution/scheduling.

\item GUI unit testing within Eclipse

\end{denseenumerate}



\end{document}



