\documentclass[10pt,landscape]{article}
\usepackage{multicol}
\usepackage{calc}
\usepackage{ifthen}
\usepackage[landscape]{geometry} 
\usepackage{times}
\usepackage{listings}

\usepackage{termrefr}
\usepackage{longtable}






%\usepackage{myalltt}%%%
% If you're reading this, be prepared for confusion.  Making this was
% a learning experience for me, and it shows.  Much of the placement
% was hacked in; if you make it better, let me know...


% 2008-04
% Changed page margin code to use the geometry package. Also added code for
% conditional page margins, depending on paper size. Thanks to Uwe Ziegenhagen
% for the suggestions.
% 2006-08
% Made changes based on suggestions from Gene Cooperman. <gene at ccs.neu.edu>


% To Do:
% \listoffigures \listoftables
% \setcounter{secnumdepth}{0}


% This sets page margins to .5 inch if using letter paper, and to 1cm
% if using A4 paper. (This probably isn't strictly necessary.)
% If using another size paper, use default 1cm margins.
\ifthenelse{\lengthtest { \paperwidth = 11in}}
	{ \geometry{top=.5in,left=.5in,right=.5in,bottom=.5in} }
	{\ifthenelse{ \lengthtest{ \paperwidth = 297mm}}
		{\geometry{top=1cm,left=1cm,right=1cm,bottom=1cm} }
		{\geometry{top=1cm,left=1cm,right=1cm,bottom=1cm} }
	}

% Turn off header and footer
\pagestyle{empty}
 

% Redefine section commands to use less space
\makeatletter
\renewcommand{\section}{\@startsection{section}{1}{0mm}%
                                {-1ex plus -.5ex minus -.2ex}%
                                {0.5ex plus .2ex}%x
                                {\normalfont\large\bfseries}}
\renewcommand{\subsection}{\@startsection{subsection}{2}{0mm}%
                                {-1explus -.5ex minus -.2ex}%
                                {0.5ex plus .2ex}%
                                {\normalfont\normalsize\bfseries}}
\renewcommand{\subsubsection}{\@startsection{subsubsection}{3}{0mm}%
                                {-1ex plus -.5ex minus -.2ex}%
                                {1ex plus .2ex}%
                                {\normalfont\small\bfseries}}
\makeatother



% Don't print section numbers
\setcounter{secnumdepth}{0}


\setlength{\parindent}{0pt}
\setlength{\parskip}{0pt plus 0.5ex}
%---------------------
\renewcommand{\thefootnote}{\alph{footnote}}
\newcommand{\keyw}[1]{{\bf\texttt #1}}
\newcommand{\kw}[1]{{\bf\ttfamily #1}}

% -----------------------------------------------------------------------

\begin{document}

\raggedright
\footnotesize
\begin{multicols}{2}


% multicol parameters
% These lengths are set only within the two main columns
%\setlength{\columnseprule}{0.25pt}
\setlength{\premulticols}{1pt}
\setlength{\postmulticols}{1pt}
\setlength{\multicolsep}{1pt}
\setlength{\columnsep}{2pt}

\begin{center}
     \Large{Quick Maven} \\
\end{center}
\begin{minipage}[c]{1\linewidth}
\section{General}
Maven is a software tool for Java project management and build automation. It is similar in functionality to the Apache Ant tool, but is based on different concepts.\\
In this sheet a number of commands used for Overture are described. They are all terminal commands and executed in a terminal in the following form.\\
\begin{center}

\vfil
\fbox{\keyw{mvn} \textbf{option} \textbf{goal}}\\
\vfil
\end{center}
Where a maven \textbf{goal} is the desired action which should be performed on the maven project where the terminal is currently active.

\subsection{Build Lifecycles}
The build lifecycle is a list of named \textit{phases} which can be used to give order to \textbf{goal} execution. This means that a goal of a plug-in can be bound to a specific \textit{phase} thus called when this \textit{phase} is reached in the build:

\begin{itemize}
	\item process-resources
	\item compile
	\item process-test-resources
	\item test-compile
	\item test
	\item package
	\item install
	\item deploy
\end{itemize}

\subsection{Basic goals}

\begin{tabular}{lp{8cm}}
	\keyw{compile}	& Compiles the project and all sub projects\\
	\keyw{test}		& Runs compile and then tests\\
	\keyw{package}	& Build a jar containing the compiled resources of the project.\\
	\keyw{install}	& Installs the jar of the project in the local repository including it's pom so other plug-ins which depends on it just can add this to their classpath.\\
	\keyw{assembly:assembly}	& Build an executable jar including all depended plug-ins.\footnote{Not a default goal it comes from a assembly plug-in where the manifest needs to be specified}
\end{tabular}

\subsection{Options}
\begin{tabular}{lp{8cm}}
	\keyw{-X}	& Show debug info\\
	\keyw{-o}	& Run off line. This speeds up the process a lot if you already have downloaded all depended plug-ins to the local repository.\\
	\keyw{-N}	& Do not run recursive. (Do not run sub projects)\\
	\keyw{-fae}	& Fail at end. This is use full if running a root project. Allow all non-impacted builds to complete.\\
	\keyw{-Dmaven.test.skip=true}	& Skip all testing
\end{tabular}
\end{minipage}
\begin{minipage}[c]{1\linewidth}
\section{Eclipse}

\subsection{maven-eclipse-plugin Goals}
\begin{tabular}{lp{8cm}}
	\keyw{eclipse:eclipse}	& Creates Eclipse project files for a project.\\
	\keyw{eclipse:clean}		& Deletes all Eclipse files from a project.
\end{tabular}


\subsection{maven-psteclipse-plugin Goals}

\begin{tabular}{lp{8cm}}
	\keyw{psteclipse:eclipse-plugin}	& Generates Eclipse files based on the packing of the pom:
	\begin{tabular}{lp{5cm}}
		\textit{source-plugin}& Creates the Eclipse structure if missing.\\
		\textit{binary-plugin}& Creates the Eclipse structure if missing and a lib folder with the jars from all depended jar files. It also generated the Manifest with all classes from the jar files exported.\footnote{It can also be made to import a number of packages.}\\
	\end{tabular}
\end{tabular}
\end{minipage}




%\begin{minipage}[c]{1\linewidth}
%
%\end{minipage}
%\vfill
%
%\scriptsize
%
%
%
\end{multicols}

\end{document}
