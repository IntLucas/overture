\documentclass[]{article}
\usepackage[draft]{fixme}

\title{Implementing the Overture Automatic Proof System for VDM}

\author{Miguel Alexandre Ferreira\\
        Software Improvement Group, The Netherlands\\ 
		\texttt{m.ferreira@sig.nl}}

\date{\today}

\begin{document}

\maketitle
\begin{abstract}
Sander Vermolen has produced a VDM to HOL translator prototype together with HOL tactics that allow VDM proof obligations to be discharged in the theorem prover HOL.
Vermolen's VDM++ prototype is complete enough to discharge proof obligations arising from functional VDM++ models.
However the prototype's performance is very limited as it can only be executed within a VDM++ interpreter.
Through the VDM++ to Java code generator tool, provided by the VDMTools, the prototype was implemented in Java allowing for its integration in the Overture Automatic Proof System.

The Automatic Proof System is a Java program that promotes the interoperation between a VDM proof obligation generation tool, the VDM to HOL translator, and the theorem prover HOL.
The current paper reports on the challenges and achievements of the Automatic Proof System's implementation, as part of the Overture Tool framework.
\end{abstract}

\section{Introduction}
\label{sec:introduction}

The Overture initiative is a platform that enables researchers, students and practitioners to experiment with software modelling languages and tools.
Although the initiative has mainly focused in VDM and its three dialects, due to its open-source nature, everyone is welcome to contribute with tools and extensions to other languages.
Through Overture, extensions to the different VDM dialects and supporting tools have been proposed, analysed, tested and finally transferred to industrial settings, namely to the VDMTools~\cite{VDMTools}.
However Overture's contribution has gone further than that, as new tools have been developed that supersede the functionality provided by the VDMTools.

One of the tools that adds to the capabilities of the VDMTools is the Automatic Proof Support (APS), which is capable of discharging proof obligations (POs) arising from functional VDM models using the theorem prover HOL~\cite{HOL}.
Due to VDM's formal semantics it is possible to analyse a VDM model and pinpoint the locations where inconsistencies might occur.
Such inconsistencies can arise from invariants violations, miss usage of partial functions and mappings, etc.
Furthermore, besides pinpointing possible inconsistencies, it is also possible to generate verification conditions that if be proven true assure the model's consistency.
In a VDM context, these verification conditions are deemed POs.

The APS was designed by Sander Verm\"olen during his MSc project~\cite{Sander}.
As deliverables from the project, Verm\"olen produced a VDM++ model of a tool that translates VDM to HOL syntax (the Overture VDM-to-HOL Translator); plus a set of lemmas, which he identified as useful in a VDM context, together with a set of tactics to automate the proofs.

The implementation described in this paper follows Verm\"olen's design as truly as possible, but also adds to it whenever found necessary.

In the next section Verm\"olen's architecture for APS is described together with the discussion of the difficulties to implement it, and possible solutions.
Section~\ref{sec:implementation} covers the actual implementation issues both at architectural and code level.
Future work is presented in Section~\ref{sec:future_work}, and the paper terminates with some conclusions in Section~\ref{sec:conclusion}.




\section{Architecture}
\label{sec:intended_architecure}

%Sander's architecture \ldots
%Description of components \ldots
%
%Automated proof achieved in three steps: preparation, translation, proof.
%
%Preparation:
%- convert model's concrete syntax in abstract syntax
%- generate POs from model's AST. POs are already in AST format
%Translation:
%- VDM AST to HOL AST (model + POs)
%- combine HOL AST model with HOL AST POs
%Proof:
%- HOL code generation
%- HOL tactics selection

The APS workflow as originally designed by Verm\"olen divides the system's operation in three sequential steps: preparation, translation, and proof.

\begin{description}
  \item[Preparation.] Preparation is the step that converts a VDM model concrete syntax into the abstract syntax format expected by the translator tool.
It's also in the preparation step that the POs, arising from the model, get generated by a PO generation (POG) tool.
The POG analyses the model's abstract representation and produces the PO expressions in the same abstract format.

  \item[Translation.] Given both model and POs, in abstract representation, in the translation step an abstract representation of an equivalent HOL model is created by translating both in separate, followed by their combination.

  \item[Proof.] In last step in the workflow, the proof, the concrete HOL syntax is generated from its abstract representation and a proof tactic is selected for each PO, according with its type~\footnote{There are several types of PO. For more detailed information see~\ref{VDMToolsManual}}. The last activity in the workflow is the actual proof carried out in the theorem prover.
\end{description}

Following from the described workflow, the architecture of the system is composed of three high level components, one per each step as depicted in Figure~\ref{fig:sander_arch}.
Each of these high level components can be decomposed in sub components, which may be part of the APS itself or external tools that need to be integrated.

\fixme[inline]{Add breakdown pictures for each component.}

\subsection{Difficulties and Solutions}
\label{sub:architectural_difficulties}

The architecture, as described up until now, can be regarded as the ideal architecture for a scenario where all external tools provide the necessary features.
However, his is not yet the case and two adjustments have to be made.

% There is no POG that generates AST \ldots
The first thing to notice is that, although Verm\"olen's design makes perfect sense, there is no available combination of POG tool and public VDM AST format to be used in preparation.
There are two POG tools that can be used both belonging to tool sets that are implemented in monolithic packages, namely the commercial VDMTools and open source VDMJ.
Moreover, the translation tool of the translation step expects both the model and POs to be in the Overture Modelling Language (OML) AST representation, and neither of the POG tools available expects the model to be in this format, let alone generating POs in OML AST.
Both these POG tools expect the models to be in VDM concrete syntax and produce textual, human readable, representations of the POs.
This mismatch between the available tools and the intended architecture makes it necessary to adjust the architecture by introducing the Overture Parser as part of the preparation step, as depicted in Figure~\ref{fig:archi_with_ovt_parser}.
In this arrangement the system uses the Overture Parser to generate OML AST representations for the model and each POs individually.

% The HOL code generation and tactics selection is done by the translator tool
The second adjustment needed is related with the proof step, more specifically with the HOL concrete syntax generation and tactics selection.
Both of these activities are performed by the translation tool, and therefore they are performed in the translation step instead of the proof step.
However, from the experience gathered by the use of the APS in some case studies~\cite{MiguelMSc,SBMF09paper}, the tactics selection as done by the translation tool is suboptimal as the same tactic gets chosen for each and every type of PO.
This issue was not addressed in the first version of the APS and is left as future work (see Section~\ref{sec:future_work}).

\section{Implementation}
\label{sec:implementation}

The APS is a component of the Overture Tool, and therefore must abide by the development standards of the Overture Initiative.
The Overture Tool is a formal modelling and verification tool suite that is completely modular, and integrates in the Eclipse platform through several plugins.
The main implementation language is Java and the APS will be no exception to that.

Figure~\ref{fig:??} clearly shows that the APS is nothing more than a tool-chain of components that contribute to the system's goal.
All the computations that are in fact implemented in the APS are either to promote interoperation fo the components or to increase usability.

%Integration of components \ldots
As far as the integration of components goes the implementation uses several ways to achieve it.
Both Overture Parser and the Overture VDM-to-HOL Translator have a publicly available Java API and therefore can be seemingly integrated with the APS.
As for the generation of POs, the VDMTools was chosen to perform the underlying tasks and because its CORBA interface doesn't provided the necessary API the interoperation is done though the Command Line Interface (CLI).
The choice could have been VDMJ as it offers a Java API that could be called directly from the APS code.
However, the VDMJ was still under development at the time of the implementation and the VDMTools is much more mature and haevily tested.
This doesn't mean that the APS relies exclusively on the VDMTools because the PO generation and parsing of their textual representations are abstracted by Java interfaces that provide the necessary decoupling from implementations.
Although the VDMJ is not yet supported as a POG tool for the APS, the system is prepared to allow its integration in the tool-chain (see Section~\ref{sec:future_work} for more information on this subject).


%Usability \ldots

\subsection{Difficulties and Solutions}
\label{sub:implementation_difficulties}

Ad-hoc parsing of POs \ldots

Java child process bug \ldots

\section{Future Work}
\label{sec:future_work}

Eclipse \ldots

\ldots

\section{Conclusion}
\label{sec:conclusion}

\ldots

\end{document}