%
% Table of conent frame
%
\begin{frame}[c]
	\titlepage
\end{frame}

\begin{frame}[c]
  \frametitle{Outline}
  \tableofcontents %[pausesections] %[current,currentsubsection]
%  \tableofcontents[current]
\end{frame}


%#######################################################################
\section{Overture}
%-----------------------------------------------------------------------
\frame
{
  \frametitle{The Overture project}

\begin{block}<+->{Mission}
	Overture's mission is twofold: 
  \begin{itemize}
  		\item to provide an industrial-strength tool that supports the use of precise abstract models in software development, and 
  		\item to foster an environment that allows researchers and other interested parties to experiment with modifications and extensions to the tool.      

  \end{itemize}
\end{block}

The Overture tools are being developed by volunteers, researchers and students.
}



%-----------------------------------------------------------------------
\subsection{Sites}
%-----------------------------------------------------------------------
\frame
{
  \frametitle{Sites}


\begin{center}
  \begin{itemize}
	%\itemsep=1cm
  		\item Main page (Twiki): www.overturetool.org
  		\item Download page: https://sf.net/projects/overture/
  		\item Documentation: http://overture.sf.net/maven/doc/
  \end{itemize}
\end{center}
}
%-----------------------------------------------------------------------
\subsection{Development environment}
%-----------------------------------------------------------------------
\frame
{
  \frametitle{Development environment}


\begin{block}{Base environment}
  \begin{itemize}
	%\itemsep=1cm
  		\item Maven 2 + overture repositories
  		\item Java SDK 1.5 or greater
  \end{itemize}
\end{block}

\begin{block}{GUI environment}
  \begin{itemize}
	%\itemsep=1cm
  		\item Eclipse 3.5.1 Classic
  		\item DLTK 1.0
  \end{itemize}
\end{block}
}

%#######################################################################
\section{Contributing}
%-----------------------------------------------------------------------
%
% Outline
%
\begin{frame}
  \frametitle{Outline}
  \tableofcontents[current]
\end{frame}


\frame
{
  \frametitle{How to contribute}


Our source is located in Source Forge and is frealy avaliable to download.
Request to join your group and give us an idea of what you want to contribute.
}

%-----------------------------------------------------------------------
\subsection{Structure}
%-----------------------------------------------------------------------
\frame
{
  \frametitle{Project structure}

\begin{columns}
\begin{column}{5cm}
	\begin{block}{Core}
		\begin{itemize}
			\item AST
			\item Parser
			\item VDMJ Interpeter
			\item Proofsupport
			\item Traces
			\item Umltrans
		\end{itemize}
	 \end{block}
\end{column}
\begin{column}{5cm}
\pause
	\begin{block}{IDE}
		\begin{itemize}
			\item Base Editor
			\item VDM-SL editor
			\item VDM-PP editor
			\item VDM-RT editor
			\item *generated* plugins
			\item *plugins* GUI plugins
			\item ...
		\end{itemize}
		
	\end{block}
\end{column}
\end{columns}
\pause
\begin{block}{Tools}
	\begin{itemize}
		\item Maven pst. Enables Maven-Eclipse integration (generated plugins)
		\item ASTGen - AST generation
		\item VDMT - VDM Tools integration
	\end{itemize}
\end{block}

}


\frame
{
  \frametitle{Project structure - Eclipse plug-ins}

The Overture Eclipse plug-ins are structures into two groups

\begin{columns}
\begin{column}{5cm}
	\begin{block}{Generated}
	\begin{itemize}
	%\itemsep=1cm
		\item VDMJ Interpeter
		\item Proofsupport
		\item Traces
		\item Umltrans
		
	 \end{itemize}
	\end{block}
\end{column}
\begin{column}{5cm}
	\pause
	\begin{block}{GUI plug-ins}
	\begin{itemize}
		\item Combinatorial Testing
		\item UML Transformation
	\end{itemize}
	\end{block}
\end{column}
\end{columns}

}


\frame
{
  \frametitle{Check out}

Subversion on Source Forge:
\begin{block}{SVN}
\small \texttt{svn checkout https://overture.svn.sourceforge.net/svnroot/overture/trunk overture}
\end{block}

Prepare source for development after check-out
\begin{block}{Prepare source}
\begin{itemize}
	\item Windows: \small \texttt{make.bat install}
	\item Linux and Mac: \small \texttt{sh make.sh install}
\end{itemize}
\end{block}

}

%#######################################################################
\section{Example}
%-----------------------------------------------------------------------
%
% Outline
%
\begin{frame}
  \frametitle{Outline}
  \tableofcontents[current]
\end{frame}


%-----------------------------------------------------------------------
\subsection{Core}
%-----------------------------------------------------------------------
\frame
{
  \frametitle{Adding a core component}


\begin{center}
 
\end{center}
}

%-----------------------------------------------------------------------
\subsection{Eclipse}
%-----------------------------------------------------------------------


\frame
{
  \frametitle{How to create a GUI}


\begin{center}
 
\end{center}
}

\frame
{
  \frametitle{Wrap core component into a plug-in}


\begin{center}
 
\end{center}
}

\frame
{
  \frametitle{Creating Eclipse plug-in}


\begin{center}
 
\end{center}
}



\begin{frame}[plain,c]
  \begin{center}
	\LARGE \structure{Thank you!}

	\vspace{2cm}
	\href{www.overturetool.org}{www.overturetool.org}
\end{center}
\end{frame}
