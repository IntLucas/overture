%
% Table of conent frame
%
\begin{frame}[c]
	\titlepage
\end{frame}

\begin{frame}[c]
  \frametitle{Outline}
  \tableofcontents %[pausesections] %[current,currentsubsection]
%  \tableofcontents[current]
\end{frame}


%#######################################################################
\section{Overture}
%-----------------------------------------------------------------------
\frame
{
  \frametitle{The Overture project}

\begin{block}<+->{Mission}
	Overture's mission is twofold: 
  \begin{itemize}
  		\item to provide an industrial-strength tool that supports the use of precise abstract models in software development, and 
  		\item to foster an environment that allows researchers and other interested parties to experiment with modifications and extensions to the tool.      

  \end{itemize}
\end{block}

The Overture tools are being developed by volunteers, researchers and students.
}



%-----------------------------------------------------------------------
\subsection{Sites}
%-----------------------------------------------------------------------
\frame
{
  \frametitle{Sites}


\begin{center}
  \begin{itemize}
	%\itemsep=1cm
  		\item Main page (Twiki): \url{www.overturetool.org}
  		\item Download page: \url{http://sf.net/projects/overture}
  		\item Documentation: \url{http://overture.sf.net/maven/doc}
  \end{itemize}
\end{center}
}
%-----------------------------------------------------------------------
\subsection{Development environment}
%-----------------------------------------------------------------------
\frame
{
  \frametitle{Development environment}


\begin{block}<1->{Base environment}
  \begin{itemize}
	%\itemsep=1cm
  		\item Maven 2 + Overture repositories
  		\item Java SDK 1.5 or newer
  \end{itemize}
\end{block}

\begin{block}<2->{GUI environment}
  \begin{itemize}
	%\itemsep=1cm
  		\item Eclipse 3.5.1 Classic
		\item Maven Integration for Eclipse
  		\item DLTK 1.0
  \end{itemize}
\end{block}
}

%#######################################################################
\section{Contributing}
%-----------------------------------------------------------------------
%
% Outline
%
\begin{frame}
  \frametitle{Outline}
  \tableofcontents[current]
\end{frame}

%-----------------------------------------------------------------------
\subsection{Contributing}
%-----------------------------------------------------------------------

\frame
{
  \frametitle{How to contribute}

\begin{itemize}
\item Join us at Source Forge
	\begin{itemize}
	\item Project name: \textbf{Overture Tool}
	\end{itemize}
\item Send us a short description of the project
\item Check out the source
\end{itemize}

}

%-----------------------------------------------------------------------
\subsection{Project structure}
%-----------------------------------------------------------------------
\frame
{
  \frametitle{Project structure}

\begin{columns}
\begin{column}{5cm}
	\begin{block}{Core}
		\begin{itemize}
			\item AST
			\item Parser
			\item VDMJ Interpreter
			\item Proofsupport
			\item Traces
			\item Umltrans
		\end{itemize}
	 \end{block}
\end{column}
\begin{column}{5cm}
\pause
	\begin{block}{IDE}
		\begin{itemize}
			\item Base Editor
			\item VDM-SL editor
			\item VDM-PP editor
			\item VDM-RT editor
			\item \textbf{Generated} plug-ins
			\item GUI plug-ins
		\end{itemize}
		
	\end{block}
\end{column}
\end{columns}
\pause
\begin{block}{Tools}
	Maven pst, ASTGen, VDMT
\end{block}

}

\note
{
\begin{itemize}
		\item Maven pst. Enables Maven-Eclipse integration (generated plugins)
		\item ASTGen - AST generation
		\item VDMT - VDM Tools integration
\end{itemize}

}

\frame
{
  \frametitle{IDE packages}

\begin{center}
	\includegraphics<1->[width=0.8\textwidth]{images/idepackages.png}%
\end{center}
}


\begin{frame}[fragile]

  \frametitle{Project structure - Eclipse plug-ins}

The Overture Eclipse plug-ins are structured into two groups

\vspace{1cm}
\begin{center}


\tikzstyle{int}=[draw, fill=blue!20, minimum size=2em]
\tikzstyle{init} = [pin edge={to-,thin,black}]
\begin{tikzpicture}[node distance=3cm,auto,>=latex']
    \node [int] (a) {Core};
    \node (b) [left of=a,node distance=1cm, coordinate] {};
    \node [int] (c) [right of=a] {Generated plug-in};
    \node [coordinate] (end) [right of=c, node distance=1cm]{};
	\node [int] (d) [right of=end] {GUI plug-in};
%\node [coordinate1] (end1) [right of=d, node distance=1cm]{};
    \path[->] (b) edge node {} (a);
    \path[->] (a) edge node {} (c);
    \path[->] (c) edge node {} (d);
%    \draw[->] (d) edge node {} (end1) ;
\end{tikzpicture}
\end{center}

\begin{columns}
\begin{column}{5cm}
	\begin{block}<2->{Generated}
	\begin{itemize}
	%\itemsep=1cm
		\item VDMJ Interpreter
		\item Proofsupport
		\item Traces
		\item Umltrans
		
	 \end{itemize}
	\end{block}
\end{column}
\begin{column}{5cm}
	\pause
	\begin{block}<2->{GUI plug-ins}
	\begin{itemize}
		\item Combinatorial Testing
		\item UML Transformation
	\end{itemize}
	\end{block}
\end{column}
\end{columns}


\end{frame}

%-----------------------------------------------------------------------
\subsection{Getting the sources}
%-----------------------------------------------------------------------

\frame
{
  \frametitle{Check out}

Subversion on Source Forge:
\begin{block}{SVN}
\small \texttt{svn checkout https://overture.svn.sourceforge.net/svnroot/overture/trunk overture}
\end{block}

Prepare source for development after check-out
\begin{block}{Prepare source}
\begin{itemize}
	\item Windows: \small \texttt{make.bat install}
	\item Linux and Mac: \small \texttt{./make.sh install}
\end{itemize}
\end{block}

}

\note
{
\begin{block}{Use full maven goals}
\begin{itemize}
	\item install - installs a plugin in the local repository
	\item eclipse:eclipse - generates eclipse project files (needed before eclipse import)
	\item psteclipse:eclipse-plugin - fetched jar files from local repository and generated binary plugins from e.g. the Manifest files
\end{itemize}
\end{block}
}

%#######################################################################
\section{Example}
%-----------------------------------------------------------------------
%
% Outline
%
\begin{frame}
  \frametitle{Outline}
  \tableofcontents[current]
\end{frame}


%-----------------------------------------------------------------------
\subsection{Core}
%-----------------------------------------------------------------------
\begin{frame}[fragile]
  \frametitle{Adding a core component}
  
  Navigate to \ttfamily{trunk/core}
   
  \lstset{language=XML,
		frame=ltrb,
		framesep=5pt,
		showtabs=true,
		basicstyle=\ttfamily\tiny,
		keywordstyle=\ttfamily,
		identifierstyle=\ttfamily\color{blue}\bfseries,
		commentstyle=\color{Brown},
		stringstyle=\ttfamily,
		showstringspaces=ture}
  \begin{lstlisting}
  mvn archetype:create -DgroupId=org.overturetool -DartifactId=umltrans
  \end{lstlisting}

\begin{columns}
\begin{column}{5cm}
	\begin{itemize}
	%\itemsep=1cm 
		\item Create pom "umltrans"
		\begin{itemize}
			\item Name
			\item Group
			\item Version
			\item Description
			\item Depended artifacts
		\end{itemize}
		
	 \end{itemize}
\end{column}
\begin{column}{6cm}
	\pause
	\lstset{language=XML,
		frame=ltrb,
		framesep=5pt,
		showtabs=true,
		basicstyle=\ttfamily\tiny,
		keywordstyle=\ttfamily,
		identifierstyle=\ttfamily\color{blue}\bfseries,
		commentstyle=\color{Brown},
		stringstyle=\ttfamily,
		showstringspaces=ture}
	\begin{lstlisting}
<project>
  <parent>
    <groupId>org.overturetool</groupId>
    <artifactId>core</artifactId>
    <version>2.0.0</version>
  </parent>
  <groupId>org.overturetool</groupId>
  <artifactId>umltrans</artifactId>
  <version>2.0.0</version>
  <name>Bi-directional UML translator</name>
  <description>A descriping text
  </description>
  <dependencies>
    <dependency>
      <groupId>org.overturetool</groupId>
      <artifactId>stdlib</artifactId>
      <version>2.0.0</version>
    </dependency>
  </dependencies>
</project>
	\end{lstlisting}
\end{column}
\end{columns}


\end{frame}

%-----------------------------------------------------------------------
\subsection{Eclipse}
%-----------------------------------------------------------------------


\frame
{
  \frametitle{How to create a GUI}
\begin{enumerate}
	\item Install core artifact
	\item Create core artifact plug-in wrapper
	\item Run maven psteclipse:eclipse.plugin to create the eclipse plug-in
	\item Create the gui plug-in
	\item Reference the wrapped core plug-in
\end{enumerate}


}

\begin{frame}[fragile]
  \frametitle{Wrap core component into a plug-in}

Navigate to \ttfamily{ide/generated/}
  \lstset{language=XML,
		frame=ltrb,
		framesep=5pt,
		showtabs=true,
		basicstyle=\ttfamily\tiny,
		keywordstyle=\ttfamily,
		identifierstyle=\ttfamily\color{blue}\bfseries,
		commentstyle=\color{Brown},
		stringstyle=\ttfamily,
		showstringspaces=ture}
  \begin{lstlisting}
  mvn archetype:create 
		-DgroupId=org.overture.ide.generated
		-DartifactId=org.overture.ide.generated.umltrans 
  \end{lstlisting}



\onslide<2->{ 
\begin{tikzpicture}
[
	overlay,
	xscale	= 1,	% to scale horizontally everything but the text
	yscale	= 1,	% to scale vertically everything but the text
]
	\node[fill=green!20,text height=0.2ex,text width=30ex, text depth=0.25ex]	at	(2.5,-2.2) (t1){ };

\end{tikzpicture}
}

\lstset{language=XML,
		frame=ltrb,
		framesep=5pt,
		showtabs=true,
		basicstyle=\ttfamily\tiny,
		keywordstyle=\ttfamily,
		identifierstyle=\ttfamily\color{blue}\bfseries,
		commentstyle=\color{Brown},
		stringstyle=\ttfamily,
		showstringspaces=ture}
\begin{lstlisting}
<project>
  ...
  <groupId>org.overture.ide.generated</groupId>
  <artifactId>org.overture.ide.generated.umltrans</artifactId>
  <name>org.overture.ide.generated.umltrans</name>
  <version>2.0.0</version>

  <packaging>binary-plugin</packaging>

  <dependencies>
    <dependency>
      <groupId>org.overturetool</groupId>
      <artifactId>umltrans</artifactId>
      <version>2.0.0</version>
    </dependency>
  </dependencies>
</project>
\end{lstlisting}
\end{frame}



\begin{frame}[fragile]
  \frametitle{Creating Eclipse plug-in}

Navigate to ide/plugins/
  \lstset{language=XML,
		frame=ltrb,
		framesep=5pt,
		showtabs=true,
		basicstyle=\ttfamily\tiny,
		keywordstyle=\ttfamily,
		identifierstyle=\ttfamily\color{blue}\bfseries,
		commentstyle=\color{Brown},
		stringstyle=\ttfamily,
		showstringspaces=ture}
  \begin{lstlisting}
  mvn archetype:create 
	-DgroupId=org.overture.ide.plugins
	-DartifactId=org.overture.ide.plugins.umltrans 
  \end{lstlisting}


\onslide<2->{ 
\begin{tikzpicture}
[
	overlay,
	xscale	= 1,	% to scale horizontally everything but the text
	yscale	= 1,	% to scale vertically everything but the text
]
	\node[fill=green!20,text height=0.2ex,text width=28ex, text depth=0.25ex]	at	(2.8,-2.8) (t1){ };

\end{tikzpicture}
}

\lstset{language=XML,
		frame=ltrb,
		framesep=5pt,
		showtabs=true,
		basicstyle=\ttfamily\tiny,
		keywordstyle=\ttfamily,
		identifierstyle=\ttfamily\color{blue}\bfseries,
		commentstyle=\color{Brown},
		stringstyle=\ttfamily,
		showstringspaces=ture}
\begin{lstlisting}
<project>
...
  <modelVersion>4.0.0</modelVersion>
  <modelVersion>4.0.0</modelVersion>
  <groupId>org.overture.ide.plugins</groupId>
  <artifactId>org.overture.ide.plugins.umltrans</artifactId>
  <name>org.overture.ide.plugins.umltrans</name>
  <version>2.0.0</version>

  <packaging>source-plugin</packaging>

</project>
\end{lstlisting}
\end{frame}

\frame
{
  \frametitle{Import in Eclipse}
  \begin{columns}
\begin{column}{5cm}
\begin{enumerate}
	\item<1-> Run make install
	\item<2-> Import Maven Projects
	\item<4-> Update classpath and clean generated plugins
	\item<6-> Add reference in manifest to generated plugin
	\item<7-> Ready to build GUI
\end{enumerate}
\end{column}
\begin{column}{6cm}



\includegraphics<2>[width=0.9\textwidth]{images/import.png}%
\includegraphics<3>[width=\textwidth]{images/importide.png}%
\includegraphics<4>[width=\textwidth]{images/classpath.png}%
\includegraphics<5>[width=\textwidth]{images/clean.png}%
\includegraphics<6>[width=\textwidth]{images/dependencies.png}%
\includegraphics<7>[width=\textwidth]{images/eclipseready.png}%


\end{column}
\end{columns}


\begin{block}<1>{Prepare source}
\begin{itemize}
	\item Windows: \small \texttt{make.bat install}
	\item Linux and Mac: \small \texttt{./make.sh install}
\end{itemize}
\end{block}

}

\begin{frame}[plain,c]
  \begin{center}
	\LARGE \structure{Thank you!}

	\vspace{2cm}
	\href{www.overturetool.org}{www.overturetool.org}
\end{center}
\end{frame}
