\documentclass[10pt,landscape]{article}
\usepackage{multicol}
\usepackage{calc}
\usepackage{ifthen}
\usepackage[landscape]{geometry} 
\usepackage{times}
\usepackage{listings}

\usepackage{termrefr}
\usepackage{longtable}




\usepackage{Macros}

%\usepackage{myalltt}%%%
% If you're reading this, be prepared for confusion.  Making this was
% a learning experience for me, and it shows.  Much of the placement
% was hacked in; if you make it better, let me know...


% 2008-04
% Changed page margin code to use the geometry package. Also added code for
% conditional page margins, depending on paper size. Thanks to Uwe Ziegenhagen
% for the suggestions.
% 2006-08
% Made changes based on suggestions from Gene Cooperman. <gene at ccs.neu.edu>


% To Do:
% \listoffigures \listoftables
% \setcounter{secnumdepth}{0}


% This sets page margins to .5 inch if using letter paper, and to 1cm
% if using A4 paper. (This probably isn't strictly necessary.)
% If using another size paper, use default 1cm margins.
\ifthenelse{\lengthtest { \paperwidth = 11in}}
	{ \geometry{top=.5in,left=.5in,right=.5in,bottom=.5in} }
	{\ifthenelse{ \lengthtest{ \paperwidth = 297mm}}
		{\geometry{top=1cm,left=1cm,right=1cm,bottom=1cm} }
		{\geometry{top=1cm,left=1cm,right=1cm,bottom=1cm} }
	}

% Turn off header and footer
\pagestyle{empty}
 

% Redefine section commands to use less space
\makeatletter
\renewcommand{\section}{\@startsection{section}{1}{0mm}%
                                {-1ex plus -.5ex minus -.2ex}%
                                {0.5ex plus .2ex}%x
                                {\normalfont\large\bfseries}}
\renewcommand{\subsection}{\@startsection{subsection}{2}{0mm}%
                                {-1explus -.5ex minus -.2ex}%
                                {0.5ex plus .2ex}%
                                {\normalfont\normalsize\bfseries}}
\renewcommand{\subsubsection}{\@startsection{subsubsection}{3}{0mm}%
                                {-1ex plus -.5ex minus -.2ex}%
                                {1ex plus .2ex}%
                                {\normalfont\small\bfseries}}
\makeatother



% Don't print section numbers
\setcounter{secnumdepth}{0}


\setlength{\parindent}{0pt}
\setlength{\parskip}{0pt plus 0.5ex}
%---------------------
\newenvironment{TypeSemantics}{\begin{tabular}{|l|l|}}{\end{tabular}}
\def\Quote{{\mathcode`\'='047 '}}
\newcommand{\Lit}[1]{`{\tt #1}\Quote}
\newcommand{\Rule}[2]{
  \begin{quote}\begin{tabbing}
    #1\index{#1}\ \ \= = \ \ \= #2  ; %    Adds production rule to index
    
  \end{tabbing}\end{quote}
  }
\newcommand{\SETDIFF}{\texttt{\char'134}}
\newcommand{\SeqPt}[1]{\{\ #1\ \}}
\newcommand{\lfeed}{\\ \> \>}
\newcommand{\OptPt}[1]{[\ #1\ ]}
\newcommand{\dsepl}{\ $|$\ }
\newcommand{\dsep}{\\ \> $|$ \>}
\newcommand{\Lop}[1]{`{\sf #1}\Quote}
\newcommand{\blankline}{\vspace{\baselineskip}}
\newcommand{\Brack}[1]{(\ #1\ )}
\newcommand{\nmk}{\footnotemark}
\newcommand{\ntext}[1]{\footnotetext{{\bf Note: } #1}}
\newlength{\mykwlen}
\newcommand{\Keyw}[1]{\settowidth{\mykwlen}{\tt #1}\makebox[\mykwlen][l]{\sf
    #1}}
\newcommand{\keyw}[1]{{\bf\ttfamily #1}}
\newcommand{\kw}[1]{{\bf\ttfamily #1}}
\newcommand{\tool}[1]{{\sf #1}}
\newcommand{\id}[1]{{\tt #1}}
\newcommand{\metaiv}[1]{\begin{alltt}\input{#1}\end{alltt}}
\newcommand{\MAP}[2]{\kw{map }#1\kw{ to }#2}
\newcommand{\INMAP}[2]{\kw{inmap }#1\kw{ to }#2}
\newcommand{\SEQ}[1]{\kw{seq of }#1}
\newcommand{\NSEQ}[1]{\kw{seq1 of }#1}
\newcommand{\SET}[1]{\kw{set of }#1}
\newcommand{\PROD}[2]{#1 * #2}
\newcommand{\TO}[2]{#1 $\rightarrow$ #2}
\newcommand{\FUN}[2]{#1 \To #2}
\newcommand{\sindex}[1]{ }


% -----------------------------------------------------------------------

\begin{document}

\raggedright
\footnotesize
\begin{multicols}{2}


% multicol parameters
% These lengths are set only within the two main columns
%\setlength{\columnseprule}{0.25pt}
\setlength{\premulticols}{1pt}
\setlength{\postmulticols}{1pt}
\setlength{\multicolsep}{1pt}
\setlength{\columnsep}{2pt}

\begin{center}
     \Large{Quick Overview of VDM Operators} \\
\end{center}

%\section{VDM++}
\subsection{General}
\begin{description}
\item \keyw{if} \textit{predicate} \keyw{then} Expression \keyw{else} Expression

\item \keyw{cases} expression:\\
      (pattern list 1){\tt ->} Expression 1,\\
      (pattern list 2),\\
      (pattern list 3){\tt ->} Expression 2,\\
       \keyw{others {\tt ->}} Expression 3\\
   \noindent end;

\item \keyw{for all} value \keyw{in set} setOfValues\\
      \keyw{do} Expression

\item \keyw{dcl} variable \textit{\keyw{:} type} \keyw{:=} Variable creation \keyw{;}

\item \keyw{let} variable \textit{\keyw{:} type} \keyw{=} Variable creation \keyw{in} Expression

\item \keyw{let} variable \keyw{in set} setOfValues \keyw{be st} pred(variable) \keyw{in} Expression
\end{description}



\subsection{The Boolean type}\label{bool}


\begin{tabular}{|l|l|l|}\hline
    Operator       & Name       & Signature                       \\ \hline
    {\tt \keyw{not} b}& Negation   & \TO{\keyw{bool}}{\keyw{bool}} \sindex{not@\kw{not}}\sindex{negation}\\
    {\tt a \keyw{and} b}& Conjunction & \TO{\PROD{\keyw{bool}}{\keyw{bool}}}{\keyw{bool}} \sindex{and@\kw{and}} \sindex{conjunction}\\
    {\tt a \keyw{or} b}& Disjunction & \TO{\PROD{\keyw{bool}}{\keyw{bool}}}{\keyw{bool}} \sindex{or@\kw{or}} \sindex{disjunction}\\
    {\tt a => b}& Implication & \TO{\PROD{\keyw{bool}}{\keyw{bool}}}{\keyw{bool}} \sindex{implication}\\
    {\tt a <=> b}& Biimplication & \TO{\PROD{\keyw{bool}}{\keyw{bool}}}{\keyw{bool}} \sindex{biimplication}\\
    {\tt a = b} & Equality   & \TO{\PROD{\keyw{bool}}{\keyw{bool}}}{\keyw{bool}} \sindex{equality}\\
    {\tt a <> b}& Inequality & \TO{\PROD{\keyw{bool}}{\keyw{bool}}}{\keyw{bool}} \sindex{inequality}\\
    \hline
  \end{tabular}

\subsection{The numeric types}\label{subsub:numeric}



  \begin{tabular}{|l|l|l|}\hline
    Operator       & Name & Signature \\ \hline
    {\tt -x}& Unary minus & \TO{\keyw{real}}{\keyw{real}} \\
    {\tt \keyw{abs} x}& Absolute value & \TO{\keyw{real}}{\keyw{real}}\sindex{abs@\kw{abs}|textbf} \\
    {\tt x + y}& Sum    & \TO{\PROD{\keyw{real}}{\keyw{real}}}{\keyw{real}} \\
    {\tt x - y}& Difference & \TO{\PROD{\keyw{real}}{\keyw{real}}}{\keyw{real}} \\
    {\tt x * y}& Product  & \TO{\PROD{\keyw{real}}{\keyw{real}}}{\keyw{real}} \\
    {\tt x / y}& Division & \TO{\PROD{\keyw{real}}{\keyw{real}}}{\keyw{real}} \\
    {\tt x \keyw{div} y}& Integer division & \TO{\PROD{\keyw{int}}{\keyw{int}}}{\keyw{int}} \sindex{div@\kw{div}|textbf}\\
    {\tt x \keyw{mod} y}& Modulus   & \TO{\PROD{\keyw{int}}{\keyw{int}}}{\keyw{int}} \sindex{mod@\kw{mod}|textbf}\\
    {\tt x**y}& Power & \TO{\PROD{\keyw{real}}{\keyw{real}}}{\keyw{real}} \\
    {\tt x < y}& Less than & \TO{\PROD{\keyw{real}}{\keyw{real}}}{\keyw{bool}} \\
    {\tt x > y}& Greater than & \TO{\PROD{\keyw{real}}{\keyw{real}}}{\keyw{bool}} \\
    {\tt x <= y}& Less or equal & \TO{\PROD{\keyw{real}}{\keyw{real}}}{\keyw{bool}} \\
    {\tt x >= y}& Greater or equal & \TO{\PROD{\keyw{real}}{\keyw{real}}}{\keyw{bool}} \\
    {\tt x = y}& Equal  & \TO{\PROD{\keyw{real}}{\keyw{real}}}{\keyw{bool}} \sindex{equality}\\
    {\tt x <> y}& Not equal & \TO{\PROD{\keyw{real}}{\keyw{real}}}{\keyw{bool}} \\
    \hline     
  \end{tabular}

\subsection{The character, quote and token types}\label{subsec:chars}



  \begin{tabular}{|l|l|l|}\hline
    Operator       & Name      & Signature \\ \hline
    {\tt c1 = c2}  & Equal     & \TO{\PROD{\keyw{char}}{\keyw{char}}}{\keyw{bool}} \sindex{equality}\\
    {\tt c1 <> c2} & Not equal & \TO{\PROD{\keyw{char}}{\keyw{char}}}{\keyw{bool}} \sindex{inequality}\\
    \hline
  \end{tabular}



\subsection{Tuple types}\label{tuples}



  \begin{tabular}{|l|l|l|} \hline
    Operator & Name & Signature \\ \hline
    {\tt t1 = t2}  & Equality   & \TO{\PROD{T}{T}}{\keyw{bool}} \sindex{equality}\\
    {\tt t1 <> t2} & Inequality & \TO{\PROD{T}{T}}{\keyw{bool}} \sindex{inequality}\\
    \hline
  \end{tabular}

\subsection{Record types}\label{records}


 

  \begin{tabular}{|l|l|l|} \hline
    Operator & Name & Signature \\ \hline
    {\tt r.i} & Field select & \TO{\PROD{A}{Id}}{Ai} \sindex{record!selector}\\
    {\tt r1 = r2} & Equality & \TO{\PROD{A}{A}}{\keyw{bool}} \sindex{equality}\\
    {\tt r1 <> r2} & Inequality & \TO{\PROD{A}{A}}{\keyw{bool}} \sindex{inequality}\\
    {\tt \keyw{is\_}A(r1)} & Is & \TO{\PROD{Id}{\texttt{MasterA}}}{\keyw{bool}} \sindex{is expression}\\
    \hline
  \end{tabular}
\vspace{1ex}



\subsection{Union and optional types}\label{unions}


  \begin{tabular}{|l|l|l|}\hline
    Operator & Name & Signature \\ \hline
    {\tt t1 = t2} & Equality & \TO{\PROD{A}{A}}{\keyw{bool}} \sindex{equality}\\
    {\tt t1 <> t2} & Inequality & \TO{\PROD{A}{A}}{\keyw{bool}} \sindex{inequality}\\
    \hline
  \end{tabular}



\subsection{Set types}
\label{sets}



  
  \begin{tabular}{|l|l|l|}\hline
    Operator & Name & Signature \\ \hline
    {\tt e \keyw{in set} s1} & Membership & \TO{\PROD{A}{\SET{A}}}{\keyw{bool}}\sindex{in set@\kw{in set}} \sindex{set!membership}\\
    {\tt e \keyw{not in set} s1} & Not membership & \TO{\PROD{A}{\SET{A}}}{\keyw{bool}} \sindex{not in set@\kw{not in set}}\\
    {\tt s1 \keyw{union} s2}& Union & \TO{\PROD{\SET{A}}{\SET{A}}}{\SET{A}} \sindex{union@\kw{union}}\sindex{set!union}\\
    {\tt s1 \keyw{inter} s2}& Intersection & \TO{\PROD{\SET{A}}{\SET{A}}}{\SET{A}} \sindex{inter@\kw{inter}} \sindex{set!intersection}\\
    {\tt s1 {\tt \char'134} s2}& Difference & \TO{\PROD{\SET{A}}{\SET{A}}}{\SET{A}}\sindex{set!difference} \\
    {\tt s1 \keyw{subset} s2}& Subset & \TO{\PROD{\SET{A}}{\SET{A}}}{\keyw{bool}} \sindex{subset@\kw{subset}}\sindex{set!subset}\\
%    {\tt s1 \keyw{psubset} s2} & Proper subset & \TO{\PROD{\SET{A}}{\SET{A}}}{\keyw{bool}} \sindex{psubset@\kw{psubset}}\sindex{set!proper subset} \\
    {\tt s1 = s2}& Equality & \TO{\PROD{\SET{A}}{\SET{A}}}{\keyw{bool}} \sindex{equality}\\
    {\tt s1 <> s2}& Inequality & \TO{\PROD{\SET{A}}{\SET{A}}}{\keyw{bool}} \sindex{inequality}\\
    {\tt \keyw{card} s1}& Cardinality & \TO{\SET{A}}{\keyw{nat}} \sindex{char@\kw{char}}\sindex{set!cardinality}\\
    {\tt \keyw{dunion} ss}& Distributed union& \TO{\SET{\SET{A}}}{\SET{A}} \sindex{dunion@\kw{dunion}}\sindex{set!union!distributed}\\
    {\tt \keyw{dinter} ss}&Distributed intersection &
    \TO{\SET{\SET{A}}}{\SET{A}} \sindex{dinter@\kw{dinter}} \sindex{set!intersection!distributed}\\
    \hline
  \end{tabular}




\subsection{Sequence types}
\label{sequences}


  
  \begin{tabular}{|l|l|l|}\hline
    Operator & Name & Signature \\ \hline
    {\tt \keyw{hd} l} & Head & \TO{\NSEQ{A}}{A} \sindex{hd@\kw{hd}}\sindex{sequence!head}\\
    {\tt \keyw{tl} l} & Tail & \TO{\NSEQ{A}}{\SEQ{A}} \sindex{tl@\kw{tl}}\sindex{sequence!tail}\\
    {\tt \keyw{len} l} & Length & \TO{\SEQ{A}}{\keyw{nat}} \sindex{len@\kw{len}}\sindex{sequence!length}\\
    {\tt \keyw{elems} l} & Elements & \TO{\SEQ{A}}{\SET{A}} \sindex{elems@\kw{elems}}\sindex{sequence!elements}\\
    {\tt \keyw{inds} l} & Indices & \TO{\SEQ{A}}{\SET{\keyw{nat1}}} \sindex{inds@\kw{inds}}\sindex{sequence!indices}\\
    {\tt l1 \char'136\ l2} & Concatenation & \TO{\PROD{(\SEQ{A})}{(\SEQ{A})}}{\SEQ{A}} \sindex{sequence!concatenation}\\
    {\tt \keyw{conc} ll} & Distributed concatenation & \TO{\SEQ{\SEQ{A}}}{\SEQ{A}}\sindex{conc@\kw{conc}}\sindex{sequence!concatenation!distributed}\\
    {\tt l ++ m} & Sequence modification & \TO{\PROD{\SEQ{A}}{\MAP{\keyw{nat}}{A}}}{\SEQ{A}}\sindex{sequence!modification}\\
    {\tt l(i)} & Sequence index & \TO{\PROD{\SEQ{A}}{\keyw{nat1}}}{A} \sindex{sequence!index}\\
    {\tt l1 = l2} & Equality & \TO{\PROD{(\SEQ{A})}{(\SEQ{A})}}{\keyw{bool}} \sindex{equality}\\
    {\tt l1 <> l2} & Inequality & \TO{\PROD{(\SEQ{A})}{(\SEQ{A})}}{\keyw{bool}} \sindex{inequality}\\
    \hline
  \end{tabular}



\subsection{Mapping types}\label{maps}



{\small  
  \begin{tabular}{|l|l|l|}\hline
    Operator & Name & Signature \\ \hline
    {\tt \keyw{dom} m} & Domain & \TO{(\MAP{A}{B})}{\SET{A}} \sindex{dom@\kw{dom}}\sindex{mapping!domain}\\
    {\tt \keyw{rng} m} & Range & \TO{(\MAP{A}{B})}{\SET{B}} \sindex{rng@\kw{rng}}\sindex{mapping!range}\\
    {\tt m1 \keyw{munion} m2} & Map union & \TO{\PROD{(\MAP{A}{B})}{(\MAP{A}{B})}}{\MAP{A}{B}} \sindex{munion@\kw{munion}}\sindex{mapping!union}\\
    {\tt m1 ++ m2} & Override & \TO{\PROD{(\MAP{A}{B})}{(\MAP{A}{B})}}{\MAP{A}{B}} \sindex{mapping!override}\\
    {\tt \keyw{merge} ms} & Distributed merge &
    \TO{\SET{(\MAP{A}{B})}}{\MAP{A}{B}}
    \sindex{merge@\kw{merge}}\sindex{mapping!distributed merge}\\
    {\tt s <: m} & Domain restrict to & \TO{\PROD{(\SET{A})}{(\MAP{A}{B})}}{\MAP{A}{B}} \sindex{mapping!domain restriction}\\
    {\tt s <-: m} & Domain restrict by & \TO{\PROD{(\SET{A})}{(\MAP{A}{B})}}{\MAP{A}{B}} \sindex{mapping!domain subtraction}\\
    {\tt m :> s} & Range restrict to & \TO{\PROD{(\MAP{A}{B})}{(\SET{B})}}{\MAP{A}{B}} \sindex{mapping!range restriction}\\
    {\tt m :-> s} & Range restrict by & \TO{\PROD{(\MAP{A}{B})}{(\SET{B})}}{\MAP{A}{B}} \sindex{mapping!range subtraction}\\
    {\tt m(d)} & Mapping apply & \TO{\PROD{(\MAP{A}{B})}{A}}{B} \sindex{mapping!application}\\
    {\tt m1 = m2} & Equality & \TO{\PROD{(\MAP{A}{B})}{(\MAP{A}{B})}}{\keyw{bool}} \sindex{equality}\\
    {\tt m1 <> m2} & Inequality & \TO{\PROD{(\MAP{A}{B})}{(\MAP{A}{B})}}{\keyw{bool}} \sindex{inequality}\\
    \hline
  \end{tabular}}


\newpage
\section{Examples}

\subsection{General}
\begin{description}
\item \keyw{if} \textit{predicate} \keyw{then} Expression \keyw{else} Expression

\item \keyw{cases} expression:\\
      (pattern list 1){\tt ->} Expression 1,\\
      (pattern list 2),\\
      (pattern list 3){\tt ->} Expression 2,\\
       \keyw{others {\tt ->}} Expression 3\\
   \noindent end;

\item \keyw{for all} value \keyw{in set} setOfValues\\
      \keyw{do} Expression

\item \keyw{dcl} variable \textit{\keyw{:} type} \keyw{:=} Variable creation \keyw{;}

\item \keyw{let} variable \textit{\keyw{:} type} \keyw{=} Variable creation \keyw{in} Expression

\item \keyw{let} variable \keyw{in set} setOfValues \keyw{be st} pred(variable) \keyw{in} Expression
\end{description}

\subsection{Comprehensions (Structure to Structure)}

\begin{minipage}[c]{1\linewidth}

\begin{lstlisting}
{element(var) | var in set setexpr & pred(var)}

[element(i) | i in set numsetexpr & pred(i)]

!\textrm{Typically:}�

[element(list(i)) | i in set inds list & pred(list(i))]

{dexpr(var) |-> rexpr(var) | var in set setexpr & pred(var)} 
\end{lstlisting}

\subsection{From Structure to Arbitrary Value}

\begin{lstlisting}
Select: set of nat -> nat
Select(s) ==
  let e in set s
  in
    e
pre s <> {}
\end{lstlisting}

\subsection{From Structure to Single Value}

\begin{lstlisting}
SumSet: set of nat -> nat
SumSet(s) ==
  if s = {}
  then 0
  else let e in set s
       in
         e + SumSet(s\{e})
measure Card
\end{lstlisting}

\subsection{From Structure to single Boolean}

\begin{lstlisting}
forall p in set setOfP & pred(p)

exists p in set setOfP & pred(p)

exists1 p in set setOfP & pred(p)
\end{lstlisting}







\end{minipage}


\begin{minipage}[c]{1\linewidth}
\vdmSpec{ClassExample.vpp}{Class Example}{VDM:ClassExample}
\end{minipage}
\vfill

\scriptsize

%Copyright \copyright\ 2008 Overture

% Should change this to be date of file, not current date.
%\verb!Revision: 1.00 , Date: 2008/10/26.!

%http://www.vdmportal.org/twiki/bin/view/Main/VDMPPoverview

\end{multicols}
\end{document}
