\documentclass{article}

\usepackage{palatino}
\usepackage{a4wide}
\usepackage{alltt}
\usepackage{makeidx}
\usepackage{vpp}
\usepackage{color}
\usepackage{vdmsl-2e}
\usepackage{longtable}
\usepackage{hyperref}
\usepackage{times}
\usepackage{listings}
\lstdefinelanguage{VDM++}
  {morekeywords={act, active, fin, req, waiting, abs, all, allsuper, always, and, answer, 
     assumption, async, atomic, be, bool, by, card, cases, char, class, comp, compose, conc, cycles,
     dcl, def, definitions, del, dinter, div, dlmodule, do, dom, dunion, duration, effect, elems, else, elseif, end,
     error, errs, exists, exists1, exit, exports, ext, floor, for, forall, from, functions, 
     general, hd, if, imports, in, inds, infer, init, inmap, input, instance, int, inter, inv, inverse, iota, is, 
     isofbaseclass, isofclass, inv, inverse, lambda, len, let, map, measure, mu,
     mutex, mod, module, nat, nat1, new, merge, 
     munion, not, of, operations, or, others, per, periodic, post, power, pre, pref, 
     private, protected, public, qsync, rd, responsibility, return, reverse,  
     sameclass, parameters, psubset, rem, renamed, rng, sel, self, seq, seq1, set, skip, specified, st, 
     start, startlist, state, static, subclass, subset, subtrace, sync, system, then, thread, 
     threadid, time, tixe, tl, to, token, traces, trap, types, undefined,
     union, uselib, using, values, 
     variables, while, with, wr, yet, RESULT, false, true, nil, periodic pref, rat, real},
   %keywordsprefix=mk\_,
   %keywordsprefix=a\_,
   %keywordsprefix=t\_,
   %keywordsprefix=w\_,
   sensitive,
   morecomment=[l]--,
   morestring=[b]",
   morestring=[b]',
  }[keywords,comments,strings]
\lstdefinelanguage{JavaCC}
  {morekeywords={options, PARSER\_BEGIN, PARSER\_END, SKIP, TOKEN},
   sensitive=false,
  }[keywords]

% define the layout for listings
\lstdefinestyle{tool}{basicstyle=\ttfamily,
                         frame=trBL, 
			 showstringspaces=false, 
			 frameround=ffff, 
			 framexleftmargin=0mm, 
			 framexrightmargin=0mm}
\lstdefinestyle{mystyle}{basicstyle=\ttfamily,
                         frame=trBL, 
%                         numbers=left, 
%			 gobble=0, 
			 showstringspaces=false, 
%			 linewidth=\textwidth, 
			 frameround=fttt, 
			 aboveskip=5mm,
			 belowskip=5mm,
			 framexleftmargin=0mm, 
			 framexrightmargin=0mm}
%\lstdefinestyle{mystyle}{basicstyle=\sffamily\small,
%			 frame=tb,
%                         numbers=left,
%			 gobble=0,
%			 showstringspaces=false,
%			 linewidth=345pt,
%			 frameround=ffff,
%			 framexleftmargin=8mm,
%			 framexrightmargin=8mm,
%			 framextopmargin=1mm,
%			 framexbottommargin=1mm,
%			 aboveskip=7mm,
%			 belowskip=5mm,
%			 xleftmargin=10mm,}

\lstset{style=mystyle}
\lstset{language=VDM++}
%\lstset{alsolanguage=Java}
% The command below enables you to escape into normal LaTeX mode inside your 
% VDM chunks by starting with a `?� character and ending with a `��
\lstset{escapeinside=?�}
%\makeatother

%\newenvironment{vpp}
%{\noindent \rule{\columnwidth}{0.5pt} \begin{alltt}}
%{\end{alltt} \rule{\columnwidth}{0.5pt}}

\newenvironment{rtinfo}{\noindent}{}

\definecolor{covered}{rgb}{0,0,0}
\definecolor{not-covered}{rgb}{1,0,0}
\newcommand{\Lit}[1]{`{\tt #1}\Quote}
\newcommand{\Rule}[2]{
  \begin{quote}\begin{tabbing}
    #1\index{#1}\ \ \= = \ \ \= #2  ; %    Adds production rule to index
    
  \end{tabbing}\end{quote}
  }
\newcommand{\SeqPt}[1]{\{\ #1\ \}}
\newcommand{\lfeed}{\\ \> \>}
\newcommand{\dsepl}{\ $|$\ }
\newcommand{\dsep}{\\ \> $|$ \>}
\newcommand{\Lop}[1]{`{\sf #1}\Quote}
\newcommand{\blankline}{\vspace{\baselineskip}}
\newcommand{\Brack}[1]{(\ #1\ )}
\newcommand{\nmk}{\footnotemark}
\newcommand{\ntext}[1]{\footnotetext{{\bf Note: } #1}}
\newlength{\kwlen}
\newcommand{\Keyw}[1]{\settowidth{\kwlen}{\tt #1}\makebox[\kwlen][l]{\sf
    #1}}
\newcommand{\keyw}[1]{{\sf #1}}
\newcommand{\id}[1]{{\tt #1}}
\newcommand{\metaiv}[1]{\begin{alltt}\input{#1}\end{alltt}}
\newcommand{\RuleTarget}[1]{\hypertarget{rule:#1}{}}
\newcommand{\Ruledef}[2]
{
  \RuleTarget{#1}\Rule{#1}{#2}%
  }
\newcommand{\Ruleref}[1]{
  \hyperlink{rule:#1}{#1}}
\newcommand{\StateDef}[1]{{\bf #1}}
\newcommand{\TypeDef}[1]{{\bf #1}}
\newcommand{\TypeOcc}[1]{{\it #1}}
\newcommand{\FuncDef}[1]{{\bf #1}}
\newcommand{\FuncOcc}[1]{{#1}}
\newcommand{\ClassDef}[1]{{\tiny #1}}

\makeindex

\begin{document}

% the title page
\title{Test Automation for Overture \\
\small{\today}}
\author{Peter Gorm Larsen, Kenneth Lausdahl, Nuno Rodrigues, Carlos Vilhena and
  Adriana Santos\\ IHA, Minho University and Danske Bank}
\date{\mbox{}}
\maketitle

\pagestyle{plain}

\tableofcontents

\pagebreak

\section{Introduction}

This document contains the VDM++ model of the test autoimation feature
for Overture enabling test case generation with combinational testing
principles. This VDM++ is then automatically translated to Java using
the Java code generator from VDMTools. The generated java code is then
integrated into the Overture environment developed on top of the
Eclipse platform. This feature is invoked from the Overture GUI from
the Eclipse SDE. 

The VDM++ model is structured such that it takes an
OmlSpecifications AST and a set of the class names that test
automation shall be done for. Global information is first extracted
using the \texttt{ExpandSpec} operation (from the \texttt{Global}
class) that extract inheritance information and global definitions
that can be used in the traces that test cases shall be generated
from. Ater this extraction the \texttt{ExpandSpecTraces} operation
from the \texttt{Expanded} class. This yields a mapping from name of
test case directory to a set of test cases (each test case being
represented as a sequence of \texttt{OmlExpression} AST's). Each of
the test cases then needs to be converted to their concrete syntax
using the visitors from the \texttt{Oml2VppVisitor} class producing a
string (in \texttt{Oml2VppVisitor.result}) that can be used as an
argument to the VDMTools interpreter. This is done for each expression
in each test case as a sequence of calls until a run-time error is
returned from the interpreter or all expressions in the test case have
been executed.  All test cases consisting of a sequence of expressions
to be executed after each other in a sequence can be displayed in the
GUI along with the generated results from VDMTools' interpreter. In
case a test case gives a run-time-error there is no reason to exetute
other test cases that have the same prefix. These are filtered away
using the \texttt{Filtering} class. The test cases that are filtered
away are thus NOT executed by the VDMTools' interpreter. At the end of
the execution statistics as well as logs from the entire execution of
the whole test suite can be produced using the \texttt{ppTestCases}
from the \texttt{Filtering} class.

This VDM++ model is structured into a number of
classes and their role can be explained as:

\begin{description}
\item[CTesting:] Currently the main class but this can probably be
  omitted entirely.
\item[DEF:] This maintains global information extracted from
  definitions in different classes that act as the context
  of the different traces. This
  information is about globally defined values and inheritance between
  classes. 
\item[Expanded:] This is the class where the core of the generation of
  test cases is placed. This includes functionality for expanding all
  kinds of trace constructs and functionality for combining different
  kinds of collections.
\item[Eval:] This is the class that contains functionality for
  evaluation of sub expressions into semantic values.
\item[SEM:] This class provides an internal representation of semantic
  values that are returned from the evaluation of expressions. In
  addition this class also have functionality for moving semantic
  values into \texttt{OmlExpression} ASTs.
\item[Filtering:] This class is responsible for filtering the test
  cases that have the same prefix as other test cases that have
  returned a run-time error away, such that they don't need to
  executed. 
\item[Toolbox:] This class is responsible for interfacing to VDMTools'
  interpreter using the CORBA interface. A part of this will only be
  implemented at the Java level.
\item[Oml2VppVisitor:] This class enables the conversion from AST's
  fom OML to the conceret syntax of VDM++ as a string of characters
  that can be saved in argument files and shown to the user.
\item[StdLib:] Standard library used in an Overture context.
\end{description}

\input{def.vpp.tex}
\input{Expanded.vpp.tex}
\input{Eval.vpp.tex}
\input{sem.vpp.tex}
\input{Filtering.vpp.tex}
\input{Toolbox.vpp.tex}
%\input{Oml2Vpp.vpp.tex}
\input{Oml2VppVisitor.vpp.tex}
\input{StdLib.vpp.tex}
\input{TC.vpp.tex}

\printindex
\addcontentsline{toc}{section}{Index}

\end{document}
