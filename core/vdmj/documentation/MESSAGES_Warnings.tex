% VDMJ Error and Warning Messages
% ===============================
%
% Messages and warnings are numbered in the following blocks:
%
% 0000-0999	Internal errors
% 1000-1999	Lexical errors
% 2000-2999	Syntax errors
% 3000-3999	Type checking errors
% 4000-4999	Runtime errors
% 5000-5999	Warnings
%
% Several messages appear to be duplicates, but internally they are raised from
% different places in the code, and so the exact error number may be used to
% distinguish the cases. Some errors are followed by additional detail, such
% as the expected and actual values.

\begin{description}
\item[5000:] \texttt{"Definition <name> not used"}
\item[5001:] \texttt{"Instance variable is not initialized:\ <name>"}
\item[5002:] \texttt{"Mutex of overloaded operation"} This warning is
  provided if one defined a mutex for an operation that is defined
  using overloading. The users needs to be aware that all of the
  overloaded operations will now by synchronisation controlled by this
  constraint. 
\item[5003:] \texttt{"Permission guard of overloaded operation"}
\item[5004:] \texttt{"History expression of overloaded operation"}
\item[5005:] \texttt{"Should access member <member> from a static context"}
\item[5006:] \texttt{"Statement will not be reached"}
\item[5007:] \texttt{"Duplicate definition:\ <name>"}
\item[5008:] \texttt{"<name/location> hides <name/location>"}
\item[5009:] \texttt{"Empty set used in bind"}
\item[5010:] \texttt{"State init expression cannot be executed"}
%\item[5011:] \texttt{-}
\item[5012:] \texttt{"Recursive function has no measure"} Whenever a
  recursive function is defined the user have the possibility defining
  a measure (i.e.\ a function that takes the same parameters as the
  recursive function and returns a natural number that should decrease
  at every recursive call). If such measures are included the proof
  obligation generator can provide proof obligations that will ensure
  termination of the recursion.
%\item[5013:] \texttt{-}
\item[5014:] \texttt{"Uninitialized BUS ignored"}  This warning appears
  if one has defined a \texttt{BUS} that is not used.
\item[5015:] \texttt{"LaTeX source should start with \%comment,
  \\char'134 document, \\char'134 section or \\char'134 subsection"  }
\end{description}
